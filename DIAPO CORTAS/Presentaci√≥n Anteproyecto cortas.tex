\documentclass[11pt]{beamer}
\usetheme{CambridgeUS}
\usepackage{xcolor}
\usepackage[utf8]{inputenc}
\usepackage[spanish]{babel}
\usepackage{amsmath}
\usepackage{amsfonts}
\usepackage{amssymb}
\usepackage{graphicx}
\usepackage{ragged2e}
\setbeamertemplate{navigation symbols}{} 
\author[Kevin García - Alejandro Vargas]{Kevin García 1533173 \newline Alejandro Vargas 1525953}
\title[Anteproyecto]{Diseño y validación de muestreo de cítricos para detección de enfermedades en viveros del Valle del Cauca}


\newcommand\Wider[2][1em]{%
\makebox[\linewidth][c]{%
  \begin{minipage}{\dimexpr\textwidth+#1\relax}
  \raggedright#2
  \end{minipage}%
  }%
}


%\setbeamercovered{transparent} 
%\setbeamertemplate{navigation symbols}{} 
%\logo{} 
%\institute{} 
%\date{} 
%\subject{} 
\begin{document}
\justify
\begin{frame}
\frametitle{Titulo y proyecto}
\begin{block}{\textbf{Titulo del trabajo} }
Diseño y validación de muestreo de cítricos para detección de enfermedades en viveros del Valle del Cauca
\end{block}
\textbf{Titulo del trabajo:} Diseño y validación de muestreo de cítricos para detección de enfermedades en viveros del Valle del Cauca
\begin{itemize}
\justifying
\item Entidad encargada: AGROSAVIA (Corpoica)
~\\La Corporación Colombiana de Investigación Agropecuaria, Corpoica, es una entidad pública descentralizada de participación mixta sin ánimo de lucro, de carácter científico y técnico, cuyo objeto es desarrollar y ejecutar actividades de Investigación, Tecnología y transferir procesos de Innovación tecnológica al sector agropecuario.

~\\
\item Personal a cargo:
\begin{itemize}
\item[-]Nubia Murcia Riaño (Investigador Ph.D.)
\item[-]Mauricio Fernando Martínez (Investigador Máster)
\item[-]Elizabeth Narvaez Toro (Líder de Seguimiento y Evaluación)
\end{itemize}
\end{itemize}
\end{frame}

\begin{frame}
\frametitle{Problema y objetivos propuestos}
~\textbf{Problema estadístico:}
\begin{itemize}
\justifying
\item ¿Es posible diseñar un plan de muestreo adecuado y asequible que permita la detección temprana de estas enfermedades en los cítricos?
\item ¿Es posible estimar la cantidad de plantas infectadas en los lotes a partir del plan de muestreo diseñado?
\end{itemize}
~\textbf{Objetivos:}
\begin{itemize}
\justifying
\item Objetivo General
~\\Diseñar y validar un plan de muestreo para aceptación y rechazo de lotes de cítricos en viveros del Valle del Cauca que permita estimar la cantidad de plantas infectadas con el virus de la tristeza en el lote.
\item Objetivos Específicos
\begin{itemize}
\item[-]Proponer y diseñar diferentes tipos de muestreo tipo aceptación/rechazo para lotes de cítricos en viveros del Valle del Cauca.
\item[-]Validar los diseños muéstrales por medio de simulación teniendo en cuenta confianza y costo del muestreo.
\item[-]Estimar la cantidad de plantas infectadas con el virus de la tristeza.
\end{itemize}
\end{itemize}
\end{frame}

\begin{frame}
\frametitle{Antecedentes}
\textbf{Antecedentes Contextuales}
\begin{itemize}
\justifying
\item[1.]Vidal, E. (2010). Epidemiología de Plum pox virus y citrus tristeza virus en bloques de plantas de vivero. Métodos de control (Tesis doctoral). Universitat Politécnica de Valencia, Valencia, España.
\item[2.]Pérez,A. , Mora, B. , León, M. , García, J. \& Abad, P. (2012). Enfermedades causadas por Phytophthora en viveros de plantas ornamentales (Artículo). Universitat Politécnica de Valencia, Valencia, España. Bol. San. Veg. Plagas, 38: 143-156.
\item[3.]Parke, J. L., Knaus, B. J., Fieland, V. J., Lewis, C., and Grünwald, N. J.
2014. Phytophthora community structure analyses in Oregon nurseries
inform systems approaches to disease management. Phytopathology
104:1052-1062.
\item[4.]Caribbean food crops society. (2008).El Virus de la Tristeza de los Citricos (CTV) en Plantaciones Comerciales y Viveros
de la Repûblica Dominicana.(Plenary Session and Oral Presentations). University of Florida.
\end{itemize}
\end{frame}

\begin{frame}
\frametitle{Antecedentes}
\begin{itemize}
\item[5.]Melgara,(2018).Ocurrencia de Huanglongbing (Candidatus Liberibacter asiaticus) y su vector [Diaphorina citri Kuwayama (Hemiptera: Liviidae)] en viveros de cítricos de Masaya.(Trabajo de Graduación, MAESTRÍA EN SANIDAD VEGETAL).UNIVERSIDAD NACIONAL AGRARIA
FACULTAD DE AGRONOMIA
\end{itemize}
\textbf{Antecedente estadísticos}
\begin{itemize}
\justifying
\item[1.]Abad Corpa, E. , Leal Llopis, J. , Paredes Sidrach de Cardona, A. \& García Palomares, A. (2004). Monitorización del cumplimiento del protocolo de mantenimiento de la cateterización venosa mediante el método LQAS (Artículo). Universidad de Murcia, España. Revista electrónica semestral de enfermería.
\item[2.] Tawfik, Y. , Hoque, S. \&  Siddiqi, M. (2001). Using lot quality assurance sampling to improve immunization coverage in Bangladesh (Artículo). Bulletin of the world health organization, 79.
\end{itemize}
\end{frame}

\begin{frame}
\begin{itemize}
\item[3.]José A Andreo, Matilde Barrio, Rosa M Ramos, Miguel Torralba, Faustino Herrero \& Pedro J Saturno. (2000). Evaluación, mejora y monitorización de la adecuación de ingreso y estancia en Medicina Interna con el muestreo de aceptación de lotes (Artículo). Instituto nación de salud pública, México. ResearchGate.
\item[4.]Mark Myatt, Hans Limburg, Darwin Minassian \& Damson Katyola. (2003). Field trial of applicability of lot quality assurance sampling survey method for rapid assessment of prevalence of active tracoma. (Artículo). Bulletin of the world health organization, 81.
\end{itemize}
\end{frame}

\begin{frame}
\frametitle{Marco Teórico}
~\\El marco teórico está compuesto por algunas definiciones y teorías tanto contextuales como estadísticas necesarias en la investigación.
~\\\textbf{Marco conceptual:}
\begin{itemize}
\item Vivero
\item Lote
\item Plaga
\item Prueba
\item Virus de la tristeza
\item Huanglongbing(HLB)
\item Leprosis
\item Exocortis
\end{itemize}
\textbf{Marco Estadístico:}
\begin{itemize}
\item Muestreo
\item Muestreo probabilístico
\item Muestreo Aleatorio Simple (MAS)
\end{itemize}
\end{frame}

\begin{frame}
\frametitle{Marco Teórico}
\begin{itemize}
\item Muestreo sistemático
\item Muestreo estratificado
\item Muestreo secuencial
\item Muestreo por conglomerados
\item Muestreo no probabilístico
\item Muestreo de conveniencia
\item Muestreo arbitrario
\item Muestreo selectivo o dirigido
\item Muestreo de aceptación
\item Muestreo Hipergeométrico
\item Muestreo Binomial
\item Muestreo Poisson
\end{itemize}
\end{frame}


\end{document}
\chapter{Introducci\'{o}n}
En la introducci\'{o}n, el autor presenta y se\~{n}ala la importancia, el origen (los antecedentes te\'{o}ricos y pr\'{a}cticos), los objetivos, los alcances, las limitaciones, la metodolog\'{\i}a empleada, el significado que el estudio tiene en el avance del campo respectivo y su aplicaci\'{o}n en el \'{a}rea investigada. No debe confundirse con el resumen y se recomienda que la introducci\'{o}n tenga una extensi\'{o}n de m\'{\i}nimo 2 p\'{a}ginas y m\'{a}ximo de 4 p\'{a}ginas.\\

La presente plantilla maneja una familia de fuentes utilizada generalmente en LaTeX, conocida como Computer Modern, espec\'{\i}ficamente LMRomanM para el texto de los p\'{a}rrafos y CMU Sans Serif para los t\'{\i}tulos y subt\'{\i}tulos. Sin embargo, es posible sugerir otras fuentes tales como Garomond, Calibri, Cambria, Arial o Times New Roman, que por claridad y forma, son adecuadas para la edici\'{o}n de textos acad\'{e}micos.\\

La presente plantilla tiene como autores la Universidad Nacional de Colombia y contiene aspectos importantes de la Norma T\'{e}cnica Colombiana - NTC 1486. Esta plantilla ha sido adaptada para ser usada, en la presentacion final de los trabajos de grado de la Escuela de Estad\'{\i}stica de la Universidad del Valle.\\

Las m\'{a}rgenes, numeraci\'{o}n, tama\~{n}o de las fuentes y dem\'{a}s aspectos de formato, deben ser conservada de acuerdo con esta plantilla, la cual esta dise\~{n}ada para imprimir por lado y lado en hojas tama\~{n}o carta. Se sugiere que los encabezados cambien seg\'{u}n la secci\'{o}n del documento (para lo cual esta plantilla esta construida por secciones).\\

La redacci\'{o}n debe ser impersonal y gen\'{e}rica. La numeraci\'{o}n de las hojas sugiere que las p\'{a}ginas preliminares se realicen en n\'{u}meros romanos en may\'{u}scula y las dem\'{a}s en n\'{u}meros ar\'{a}bigos, en forma consecutiva a partir de la introducci\'{o}n que comenzar\'{a} con el n\'{u}mero 1. La cubierta y la portada no se numeran pero si se cuentan como p\'{a}ginas.\\

Se debe tener en cuenta que la Facultad de Ingenier\'{\i}a tiene reglamentada la extensi\'{o}n m\'{a}xima del trabajo de grado en 80 p\'{a}ginas, incluido anexos.\\

No se debe utilizar numeraci\'{o}n compuesta como 13A, 14B \'{o} 17 bis, entre otros, que indican superposici\'{o}n de texto en el documento. Para resaltar, puede usarse letra cursiva o negrilla. Los t\'{e}rminos de otras lenguas que aparezcan dentro del texto se escriben en cursiva.\\
\section{Planteamiento del problema}

\section{Justificación}

\section{Antecedentes}
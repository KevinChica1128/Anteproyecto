\documentclass[10pt]{beamer}
\usetheme{Frankfurt}
\usecolortheme{dolphin}
\usepackage[utf8]{inputenc}
\usepackage[spanish]{babel}
\usepackage{amsmath}
\usepackage{amsfonts}
\usepackage{amssymb}
\usepackage{graphicx}
\usepackage{ragged2e}
\setbeamertemplate{navigation symbols}{} 
\author[Kevin García - Alejandro Vargas]{Kevin García 1533173 \newline Alejandro Vargas 1525953}
\institute{Universidad del Valle}
\title{Diseño y validación de un plan de muestreo de cítricos para detección de enfermedades en viveros del Valle del Cauca}

\newenvironment{changemargin}[2]{%
  \begin{list}{}{%
    \setlength{\topsep}{0pt}%
    \setlength{\leftmargin}{#1}%
    \setlength{\rightmargin}{#2}%
    \setlength{\listparindent}{\parindent}%
    \setlength{\itemindent}{\parindent}%
    \setlength{\parsep}{\parskip}%
  }%
  \item[]}{\end{list}}%

\newenvironment<>{varblock}[2][.9\textwidth]{%
  \setlength{\textwidth}{#1}
  \begin{actionenv}#3%
    \def\insertblocktitle{#2}%
    \par%
    \usebeamertemplate{block begin}}
  {\par%
    \usebeamertemplate{block end}%
  \end{actionenv}}


\newcommand\Wider[2][3em]{%
\makebox[\linewidth][c]{%
  \begin{minipage}{\dimexpr\textwidth+#1\relax}
  \raggedright#2
  \end{minipage}%
  }%
}
%\setbeamercovered{transparent} 
%\setbeamertemplate{navigation symbols}{} 
%\logo{} 
%\institute{} 
%\date{} 
%\subject{} 
\begin{document}
\begin{frame}[plain]
\maketitle
\end{frame}

\begin{frame}{Contenido}
\tableofcontents
\end{frame}

\section{Proyecto}
\begin{frame}
\frametitle{Información del proyecto}
\begin{block}{Entidad encargada}
\begin{itemize}
\justifying
\item AGROSAVIA (Corpoica)
~\\La Corporación Colombiana de Investigación Agropecuaria, Corpoica, es una entidad pública descentralizada de participación mixta sin ánimo de lucro, de carácter científico y técnico, cuyo objeto es desarrollar y ejecutar actividades de Investigación, Tecnología y transferir procesos de Innovación tecnológica al sector agropecuario.
\end{itemize}
\end{block}
\begin{block}{Personal a cargo}
\begin{itemize}
\item[-]Nubia Murcia Riaño (Investigador Ph.D.)
\item[-]Mauricio Fernando Martínez (Investigador Máster)
\item[-]Elizabeth Narvaez Toro (Líder de Seguimiento y Evaluación)
\end{itemize}
\end{block}
\end{frame}

\section{Problema}
\subsection{Problema contextual}
\subsection{Problema estadístico}
\begin{frame}
\begin{block}{Problema contextual}
\justifying
~\\Existen diversas enfermedades en los cítricos transmitidas principalmente por injertación, vectores (organismos o insectos), y uso de herramienta, las cuales son muy dañinas para este cultivo, entre ellas, el virus de la tristeza, HLB, Leprosis y Exocortis. Estas debilitan el árbol, generando producciones escasas, y en casos avanzados puede llegar a matar el árbol. El inconveniente es que estas enfermedades son asintomáticas en edades tempranas de la planta, es decir, no podemos diferenciar a simple vista una planta infectada con una no infectada. Al sembrar una planta con alguna de estas infecciones desde el comienzo, se perderá mucho dinero invirtiendo en su mantenimiento, por lo cuál se necesita asegurar o garantizar que las plantas que van a ser sembradas y entregadas estén limpias de estas enfermedades, logrando de esta manera la producción de material certificado.
\end{block}
\begin{block}{Problema estadístico}
\begin{itemize}
\justifying
\item ¿Es posible diseñar un plan de muestreo en viveros que permita la detección temprana de estas enfermedades en los cítricos?
\end{itemize}
\end{block}
\end{frame}


\section{Justificación}
\begin{frame}
\frametitle{Justificación}
\justifying
Desde el área de la estadística se han realizado diversos estudios con lotes de cítricos criados en viveros los cuales se orientan a evaluar efectos que tienen ciertos tratamientos sobre las plantas, en los que se evalúa rendimiento, producción, crecimiento y muchos otros factores, pero muy pocos se orientan al muestreo, el cual cumple un papel muy importante ya que a partir de este se pueden tomar decisiones certeras. Con la implementación de un buen plan de muestreo en estos viveros se logrará definir si un lote puede llegar a catalogarse como “bueno” o si por el contrario hay que descartarlo como un lote infectado.
\end{frame}

\section{Objetivos}
\subsection{Objetivos general}
\subsection{Objetivos específicos}
\begin{frame}
\frametitle{Objetivos propuestos}
\begin{block}{Objetivo general}
Diseñar y validar empíricamente un plan de muestreo para aceptación y rechazo de lotes de cítricos en viveros del Valle del Cauca que permita estimar la cantidad de plantas infectadas en el lote.
\end{block}
\begin{block}{Objetivos específicos}
\begin{itemize}
\justifying
\item[-]Proponer y diseñar diferentes tipos de muestreo tipo aceptación/rechazo para lotes de cítricos en viveros del Valle del Cauca.
\item[-]Validar los diseños muéstrales por medio de estudios de simulación.
\item[-]Estimar la cantidad de plantas infectadas en el lote.
\end{itemize}
\end{block}
\end{frame}

\section{Antecedentes}
\subsection{Antecedentes contextuales}
\begin{frame}
\frametitle{Antecedentes}
\Wider{
\begin{varblock}[12cm]{Antecedentes contextuales}
\begin{itemize}
\justifying
\item[1.]Epidemiología de Plum pox virus y citrus tristeza virus en bloques de plantas de vivero. Métodos de control.(2010)\cite{AC1}
~\\El objetivo de esta tesis fue el estudio de los distintos factores que determinan la epidemiología de PPV y CTV en vivero, con el fin de establecer posibles estrategias de control. Se utilizaron tres parcelas, dos para PPV y una para el CTV en las cuales se realizó un muestreo que consistía en dividir las parcelas en bloques estadísticos imaginarios(dos bloques de plantas) y finalmente se tomaron ciertas filas de plantas.
\item[2.]Enfermedades causadas por Phytophthora en viveros de plantas ornamentales.(2012)\cite{AC2}
~\\El objetivo de este trabajo fue estudiar las enfermedades causadas por las especies de Phytophthora. Se exploraron 23 viveros de plantas ornamentales, muestreando solamente plantas sintomáticas, analizando un total de 360 plantas.
\end{itemize}
\end{varblock}
}
\end{frame}

\begin{frame}
\frametitle{Antecedentes}
\Wider{
\begin{varblock}[12cm]{Antecedentes contextuales}
\begin{itemize}
\justifying
\item[3.]Phytophthora community structure analyses in Oregon nurseries inform systems approaches to disease management.(2014)\cite{AC3}
~\\El objetivo fue describir la estructura de las comunidades de Phytophthora en 4 viveros comerciales. Se tomaron muestras de los 4 viveros cada 2 meses durante 4 años, recolectando 5 plantas de cada género en cada fecha de muestreo. Se seleccionaron plantas sintomáticas.
\item[4.]El Virus de la Tristeza de los Citricos (CTV) en Plantaciones Comerciales y Viveros de la República Dominicana.(2008)\cite{AC4}
~\\Para este proyecto se tomaron muestras de 9 viveros y dentro de cada vivero se tomaron muestras del 1\% para lotes grandes y 2\% para lotes pequeños. Los lotes estaban en rangos de entre 2000 y 40000 plantas, se determinó que el CTV incremento en más del 80\% desde los últimos 10 años.
\end{itemize}
\end{varblock}
}
\end{frame}

\begin{frame}
\frametitle{Antecedentes}
\Wider{
\begin{varblock}[12cm]{Antecedentes contextuales}
\begin{itemize}
\justifying
\item[5.]Ocurrencia de Huanglongbing (Candidatus Liberibacter asiaticus) y su vector Diaphorina citri Kuwayama (Hemiptera: Liviidae) en viveros de cítricos de Masaya.(2018)\cite{AC5}
~\\Para este experimento se tomaron plántulas de plantas en las cuales sus hojas presentaban síntomas de HLB, cada mes se tomaron 20 plántulas, al final se tenían 80 plántulas por vivero. Los resultados obtenidos fueron que para cada vivero la proporción de plantas infectadas con HLB era del 12.5\%, 25\%, 12.5\% y 25\% respectivamente.
\end{itemize}
\end{varblock}
}
\end{frame}

\subsection{Antecedentes estadísticos}
\begin{frame}
\frametitle{Antecedentes}
\Wider{
\begin{varblock}[12cm]{Antecedentes estadísticos}
\begin{itemize}
\justifying
\item[1.]Monitorización del cumplimiento del protocolo de mantenimiento de la cateterización venosa mediante el método LQAS.(2004)\cite{AE1}
~\\El objetivo fue evaluar el incumplimiento del mantenimiento de la cateterización venosa de un hospital mediante LQAS. Se evaluaron 4 criterios. Se partió de un estándar de cumplimiento del 95\% y un umbral mínimo del 85\%, un error a=5\% y un error b=20\%, se calculó un tamaño de muestra de 44 casos y el número mínimo de cumplimientos del protocolo de 39.
\item[2.]Using lot quality assurance sampling to improve immunization coverage in Bangladesh.(2001)\cite{AE2}
~\\El objetivo fue determinar las áreas de baja cobertura vacunal en cinco ciudades de Bangladesh. En el primero, se seleccionó la meta del 85\% de cobertura como umbral superior y 60\% como umbral inferior, un nivel de confianza del 80\% y se encontró que el tamaño de muestra era de 13, y el número de aceptación sería de 9 niños. En el segundo, el umbral superior fue de 60\% e inferior de 40\%, un nivel de confianza del 95\% y se encontró que el tamaño de muestra era de 16 niños.
\end{itemize}
\end{varblock}
}
\end{frame}


\begin{frame}
\frametitle{Antecedentes}
\Wider{
\begin{varblock}[12cm]{Antecedentes estadísticos}
\begin{itemize}
\justifying
\item[3.]Evaluación, mejora y monitorización de la adecuación de ingreso y estancia en Medicina Interna con el muestreo de aceptación de lotes.(2000)\cite{AE3}
~\\En este proyecto se utilizó el método de LQAS para evaluar la calidad, se tomaron muestras de pacientes en diferentes periodos primero evaluando la calidad del ingreso y estancia actual y luego tras hacer ciertas mejoras. Crearon un umbral de mala calidad que es “\% de adecuación” que funciona como regla de parada.
\item[4.]Zero Acceptance Number Sampling Plans.(2008)\cite{AE5}
~\\El autor  compara el muestreo de aceptaci\'{o}n y rechazo para atributos por el m\'{e}todo MIL-STD-105E/ANSI Z1.4 y los planes de muestreo c=0, dichas comparaciones se realizan en t\'{e}rminos de tama\~{n}os de muestra, curvas caracter\'{i}sticas de operaci\'{o}n y del AOQ(Calidad Promedio de Salida) y AOQL (M\'{a}ximo porcentaje de defectuosos esperado). Finalmente, se concluye que el m\'{e}todo propuesto en comparaci\'{o}n con el m\'{e}todo MIL-STD-105E/ANSI Z1.4 protege a\'{u}n m\'{a}s al consumidor.
\end{itemize}
\end{varblock}
}
\end{frame}

\section{Marco teórico}
\subsection{Marco conceptual}
\begin{frame}
\frametitle{Marco Teórico}
\begin{block}{Marco conceptual}
\begin{columns}
\column{.16\textwidth}
\begin{itemize}
\item Vivero
\item Lote
\end{itemize}
\column{.18\textwidth}
\begin{itemize}
\item Plaga
\item Prueba
\end{itemize}
\column{.36\textwidth}
\begin{itemize}
\item Virus de la tristeza
\item Huanglongbing(HLB)
\end{itemize}
\column{.22\textwidth}
\begin{itemize}
\item Leprosis
\item Exocortis
\end{itemize}
\end{columns}
\end{block}
\begin{block}{Marco Estadístico}
\begin{columns}
\column{.45\textwidth}
\begin{itemize}
\item Muestreo
\item Tipos de muestreo
\item Muestreo de aceptación
\item Muestreo Hipergeométrico
\item Muestreo Binomial
\item Muestreo Poisson
\item Aceppt on zero plans - AOZ
\item $\alpha$ Riesgo del productor
\end{itemize}
\column{.51\textwidth}
\begin{itemize}
\item $\beta$ Riesgo del consumidor
\item NCA (Nivel de Calidad Aceptable)
\item NCL (Nivel de Calidad Limite)
\item Número de aceptación
\item Curva característica de operación
\item Indicadores de desempeño(AOQ, AOQL, ATI)
\end{itemize}
\end{columns}
\end{block}
\end{frame}
\subsection{Marco Estadístico}

\section{Metodología}
\begin{frame}{Metodología}
\begin{block}{Visita viveros}
\justifying
Se visitaron dos viveros ubicados en Caicedonia - Valle del Cauca y en Calarcá - Quindío, los cuales presentaban una capacidad máxima aproximada de 70000 y 120000 plantas respectivamente. Se utilizó una guía de observación elaborada por nosotros, la cual nos ayudó a extraer la información necesaria para aplicar la metodología y para el respectivo proceso de simulación.
\end{block}
\begin{block}{Categorización de los lotes}
\justifying
Para este punto se categorizar\'{a}n los lotes seg\'{u}n su tama\~{n}o con el fin de conocer qu\'{e} plan aplicar para cada tama\~{n}o del lote ($N$), esto es qu\'{e} tama\~{n}o de la muestra tomar cuando el lote es peque\~{n}o, mediano o grande; las categor\'{i}as se definir\'{a}n con rangos de tama\~{n}os.
\end{block}
\begin{block}{Diseño del plan de muestreo en campo}
\justifying
Para este punto se diseñarán las propuestas metodol\'{o}gicas que se implementar\'{a}n en los viveros con el fin de que la muestra recogida no solo sea aleatoria si no tambi\'{e}n representativa.
\end{block}
\end{frame}

\begin{frame}{Metodología}
\begin{block}{Calculo de los tamaños de muestra para cada categoría}
\justifying
Se har\'{a} uso de la metodolog\'{i}a de los panes $c=0$ para calcular los tama\~{n}os de muestra correspondiente en cada categor\'{i}a previamente definida, esto con el fin de que cada rango de tama\~{n}o tenga un respectivo tama\~{n}o de muestra o porcentaje de plantas a muestrear.
\end{block}
\begin{block}{Calculo de los indicadores de desempe\~{n}o (AOQ, AOQL, ATI, OC)}
\justifying
Una vez dise\~{n}ado el plan de muestreo y definido qu\'{e} tama\~{n}os de muestra se van a utilizar, pasamos a evaluar el dise\~{n}o muestreal, esto lo hacemos a partir de los indicadores del dise\~{n}o y sus respectivas curvas. Cada una nos dir\'{a} qu\'{e} tan bueno es el muestreo, qu\'{e} riesgos se llegan a correr en cuanto a la probabilidad de aceptar lotes con $x$ cantidad de plantas infectadas, tambi\'{e}n la calidad promedio de salida y la cantidad de plantas que se deber\'{i}an inspeccionar con dicho plan si existen $x$ cantidad de plantas infectadas en el lote.
\end{block}
\end{frame}

\begin{frame}{Metodología}
\begin{block}{Categorizaci\'{o}n de los viveristas seg\'{u}n el nivel de riesgo}
\justifying
Luego de tener todo lo respectivo al muestreo, se planea crear 3 niveles de riesgo, flexible, normal y riguroso, esto para tener en cuenta en el muestreo el historial de los viveristas y ``premiar'' a los que mantengan una calidad en sus viveros alta, as\'{i} pues los viveristas comenzar\'{a}n en un nivel de riesgo normal, y pasaran a flexible o riguroso dependiendo de la cantidad de lotes ``buenos o malos'' que tengan en su historial, as\'{i} los que tengan una calidad ``buena'' en sus viveros tendr\'{a}n tama\~{n}os de muestra mas peque\~{n}os y por el contrario los que tengan calidad ``mala'' (varios lotes rechazados) pasaran a tener un  tama\~{n}o de muestra mayor, lo que implica m\'{a}s costos.
\end{block}
\begin{block}{Validaci\'{o}n de los planes por medio de simulaci\'{o}n}
\justifying
Por \'{u}ltimo se validar\'{a}n los planes dise\~{n}ados utilizando herramientas computacionales donde, se simular\'{a}n todos los posibles escenarios de los viveros con los diferentes planes para as\'{i} conocer si estos cumplen con el objetivo de detectar lotes infectados con una confianza alta.
\end{block}
\end{frame}

\bibliographystyle{plain}
  \bibliography{references}
\end{document}
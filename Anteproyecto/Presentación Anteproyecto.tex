\documentclass[11pt]{beamer}
\usetheme{CambridgeUS}
\usepackage[utf8]{inputenc}
\usepackage[spanish]{babel}
\usepackage{amsmath}
\usepackage{amsfonts}
\usepackage{amssymb}
\usepackage{graphicx}
\usepackage{ragged2e}
\setbeamertemplate{navigation symbols}{} 
\author[Kevin García - Alejandro Vargas]{Kevin García 1533173 \newline Alejandro Vargas 1525953}
\title[Anteproyecto]{Diseño y validación de muestreo de cítricos para detección de enfermedades en viveros del Valle del Cauca}


\newcommand\Wider[2][1em]{%
\makebox[\linewidth][c]{%
  \begin{minipage}{\dimexpr\textwidth+#1\relax}
  \raggedright#2
  \end{minipage}%
  }%
}


%\setbeamercovered{transparent} 
%\setbeamertemplate{navigation symbols}{} 
%\logo{} 
%\institute{} 
%\date{} 
%\subject{} 
\begin{document}
\justify
\begin{frame}
\titlepage
\end{frame}

%\begin{frame}
%\tableofcontents
%\end{frame}

\begin{frame}
\frametitle{Contenido}
\begin{itemize}
\item Información del proyecto
\item Problema Contextual
\item Problema Estadístico
\item Objetivos
\item Variable de interés
\end{itemize}
\end{frame}

\begin{frame}
\frametitle{Información del proyecto}
\begin{itemize}
\item Entidad encargada: AGROSAVIA (Corpoica)
~\\La Corporación Colombiana de Investigación Agropecuaria, Corpoica, es una entidad pública descentralizada de participación mixta sin ánimo de lucro, de carácter científico y técnico, cuyo objeto es desarrollar y ejecutar actividades de Investigación, Tecnología y transferir procesos de Innovación tecnológica al sector agropecuario.

~\\
\item Personal a cargo:
\begin{itemize}
\item[-]Nubia Murcia Riaño (Investigador Ph.D.)
\item[-]Mauricio Fernando Martínez (Investigador Máster)
\item[-]Elizabeth Narvaez Toro (Líder de Seguimiento y Evaluación)
\end{itemize}
\end{itemize}
\end{frame}


\begin{frame}
\frametitle{Problema Contextual}
~\\Existen varias especies de áfidos(pulgones) que infectan los cítricos con un virus llamado ``virus de la tristeza'', el cuál causa una de las enfermedades más dañinas de este cultivo, la cuál debilita el árbol, que da producciones escasas, llegando finalmente a matarlo en varios meses o si el virus es violento, dos o tres semanas. El inconveniente es que esta enfermedad es asintomática, es decir, no podemos diferenciar a simple vista una planta infectada con una no infectada. Al sembrar una planta con esta infección desde el comienzo, se perderá mucho dinero invirtiendo en su mantenimiento, por lo cuál se necesita asegurar o garantizar que las plantas que van a ser sembradas y entregadas estén limpias de esta enfermedad, logrando de esta manera la producción de material certificado.
\end{frame}

\begin{frame}
\frametitle{Problema Estadístico}
~\\Partiendo del problema contextual, surgen las siguientes preguntas
\begin{itemize}
\item ¿Es posible diseñar un plan de muestreo adecuado y asequible que permita la detección temprana de estas enfermedades en los cítricos?
\item ¿Es posible estimar la cantidad de plantas infectadas en los lotes a partir del plan de muestreo diseñado?
\end{itemize}
\end{frame}

\begin{frame}
\frametitle{Objetivos propuestos}
\begin{itemize}
\item Objetivo General
~\\Diseñar y validar un plan de muestreo para aceptación y rechazo de lotes de cítricos en viveros del Valle del Cauca que permita estimar la cantidad de plantas infectadas con el virus de la tristeza en el lote.
\item Objetivos Específicos
\begin{itemize}
\item[-]Proponer y diseñar diferentes tipos de muestreo tipo aceptación/rechazo para lotes de cítricos en viveros del Valle del Cauca.
\item[-]Validar los diseños muéstrales por medio de simulación teniendo en cuenta confianza y costo del muestreo.
\item[-]Estimar la cantidad de plantas infectadas con el virus de la tristeza.
\end{itemize}
\end{itemize}
\end{frame}

\begin{frame}
\frametitle{Variable de interés}
~\\Nuestra variable de interés se puede definir de la siguiente manera:
$$X= \left\{\begin{array}{cc}
             1 &   si \; la \; planta \; esta \; infectada \\
             \\ 0 &  si \; la \; planta \; no \; esta \; infectada \\
             \end{array}
   \right.$$
\end{frame}


\begin{frame}
\frametitle{Variable de interés}
~\\Para llegar a la variable de interés definida anteriormente, se lleva a cabo una prueba de laboratorio en la cuál se pueden medir hasta 45 muestras de tejido de plantas, donde se obtiene al final una coloración en el recipiente, a esta coloración se le hace una lectura visual o colorimétrica, usualmente se utiliza más la medida colorimétrica por ser mas objetiva, esta se obtiene con un valor de absorbancia, que corresponde en pocas palabras a una cuantificación de la percepción del color. Se encuentra de la siguiente manera:
\[ A_\lambda=-log_{10} \left( \frac{I}{I_0}\right) \]
~\\Donde:
~\\I es la intensidad de la luz con una longitud de onda específica $\lambda$ tras haber atravesado una muestra (intensidad de la luz transmitida).
~\\$I_0$  es la intensidad de la luz antes de entrar a la muestra (intensidad de la luz incidente).
\end{frame}

\end{document}
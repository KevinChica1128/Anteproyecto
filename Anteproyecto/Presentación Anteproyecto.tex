\documentclass[12pt]{beamer}
\usetheme{CambridgeUS}
\usepackage[utf8]{inputenc}
\usepackage[spanish]{babel}
\usepackage{amsmath}
\usepackage{amsfonts}
\usepackage{amssymb}
\usepackage{graphicx}
\usepackage{ragged2e}
\setbeamertemplate{navigation symbols}{} 
\author[Kevin García - Alejandro Vargas]{Kevin García 1533173 \newline Alejandro Vargas 1525953}
\title[Anteproyecto]{Anteproyecto}


\newcommand\Wider[2][3em]{%
\makebox[\linewidth][c]{%
  \begin{minipage}{\dimexpr\textwidth+#1\relax}
  \raggedright#2
  \end{minipage}%
  }%
}


%\setbeamercovered{transparent} 
%\setbeamertemplate{navigation symbols}{} 
%\logo{} 
%\institute{} 
%\date{} 
%\subject{} 
\begin{document}
\justify
\begin{frame}
\titlepage
\end{frame}

%\begin{frame}
%\tableofcontents
%\end{frame}
\begin{frame}
\frametitle{Titulo propuesto}
\begin{center}
~\\ DISEÑO Y VALIDACIÓN DE MUESTREO DE CÍTRICOS EN VIVEROS DEL VALLE DEL CAUCA
\end{center}
\end{frame}

\begin{frame}
\frametitle{Problema Contextual}
~\\Existen varias especies de áfidos(pulgones) que infectan los cítricos y generan una enfermedad llamada ``virus de la tristeza'', el cuál causa una de las enfermedades más dañinas de este cultivo (debilita el árbol, que da producciones escasas, llegando finalmente a matarlo en varios meses o si el virus es violento, dos o tres semanas). El inconveniente es que esta enfermedad es asintomática, es decir, no podemos diferenciar a simple vista una planta infectada con una no infectada. Al sembrar una planta con esta infección desde el comienzo, se perderá mucho dinero invirtiendo en su mantenimiento, por lo cuál se necesita asegurar o garantizar que las plantas que van a ser sembradas y entregadas estén limpias de esta enfermedad, logrando de esta manera la producción de material certificado.
\end{frame}

\begin{frame}
\frametitle{Problema Estadístico}
~\\Se debe diseñar un método de muestreo de aceptación y rechazo
\end{frame}

\begin{frame}
\frametitle{Objetivos propuestos}
\begin{itemize}
\item Objetivo General
~\\Diseñar y validar un muestreo en viveros del Valle del Cauca que permita estimar la cantidad de plantas infectadas con el virus de la tristeza en todo el lote de plantas.
\item Objetivos Específicos
\begin{itemize}
\item[-]Diseñar un muestreo adecuado de las plantas teniendo en cuenta los costos de cada evaluación de la unidad muestreada.
\item[-]Validar el muestreo por medio de simulación.
\item[-]Estimar la cantidad de plantas infectadas con el virus de la tristeza.
\end{itemize}
\end{itemize}
\end{frame}
\end{document}
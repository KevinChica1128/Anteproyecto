\documentclass[11pt]{beamer}
\usetheme{CambridgeUS}
\usepackage[utf8]{inputenc}
\usepackage[spanish]{babel}
\usepackage{amsmath}
\usepackage{amsfonts}
\usepackage{amssymb}
\usepackage{graphicx}
\usepackage{ragged2e}
\setbeamertemplate{navigation symbols}{} 
\author[Kevin García - Alejandro Vargas]{Kevin García 1533173 \newline Alejandro Vargas 1525953}
\title[Anteproyecto]{Diseño y validación de muestreo de cítricos para detección de enfermedades en viveros del Valle del Cauca}


\newcommand\Wider[2][1em]{%
\makebox[\linewidth][c]{%
  \begin{minipage}{\dimexpr\textwidth+#1\relax}
  \raggedright#2
  \end{minipage}%
  }%
}


%\setbeamercovered{transparent} 
%\setbeamertemplate{navigation symbols}{} 
%\logo{} 
%\institute{} 
%\date{} 
%\subject{} 
\begin{document}
\justify
\begin{frame}
\titlepage
\end{frame}

%\begin{frame}
%\tableofcontents
%\end{frame}

\begin{frame}
\frametitle{Contenido}
\begin{itemize}
\item Información del proyecto
\item Problema Contextual
\item Problema Estadístico
\item Objetivos
\item Variable de interés
\item Antecedentes
\begin{itemize}
\item Antecedentes Contextuales
\item Antecedentes Estadísticos
\end{itemize}
\item Marco Teórico
\begin{itemize}
\item Marco Conceptual
\item Marco Estadístico
\end{itemize}
\end{itemize}
\end{frame}

\begin{frame}
\frametitle{Información del proyecto}
\begin{itemize}
\item Entidad encargada: AGROSAVIA (Corpoica)
~\\La Corporación Colombiana de Investigación Agropecuaria, Corpoica, es una entidad pública descentralizada de participación mixta sin ánimo de lucro, de carácter científico y técnico, cuyo objeto es desarrollar y ejecutar actividades de Investigación, Tecnología y transferir procesos de Innovación tecnológica al sector agropecuario.

~\\
\item Personal a cargo:
\begin{itemize}
\item[-]Nubia Murcia Riaño (Investigador Ph.D.)
\item[-]Mauricio Fernando Martínez (Investigador Máster)
\item[-]Elizabeth Narvaez Toro (Líder de Seguimiento y Evaluación)
\end{itemize}
\end{itemize}
\end{frame}


\begin{frame}
\frametitle{Problema Contextual}
~\\Existen diversas enfermedades en los cítricos transmitidas principalmente por injertación, vectores (organismos o insectos), y uso de herramienta, las cuales son muy dañinas para este cultivo, entre ellas, el virus de la tristeza, HLB, Leprosis y Exocortis. Estas debilitan el árbol, generando producciones escasas, y en casos avanzados puede llegar a matar el árbol. El inconveniente es que estas enfermedades son asintomáticas en edades tempranas de la planta, es decir, no podemos diferenciar a simple vista una planta infectada con una no infectada. Al sembrar una planta con alguna de estas infecciones desde el comienzo, se perderá mucho dinero invirtiendo en su mantenimiento, por lo cuál se necesita asegurar o garantizar que las plantas que van a ser sembradas y entregadas estén limpias de estas enfermedades, logrando de esta manera la producción de material certificado.
\end{frame}

\begin{frame}
\frametitle{Problema Estadístico}
~\\Partiendo del problema contextual, surgen las siguientes preguntas
\begin{itemize}
\item ¿Es posible diseñar un plan de muestreo adecuado y asequible que permita la detección temprana de estas enfermedades en los cítricos?
\item ¿Es posible estimar la cantidad de plantas infectadas en los lotes a partir del plan de muestreo diseñado?
\end{itemize}
\end{frame}

\begin{frame}
\frametitle{Objetivos propuestos}
\begin{itemize}
\item Objetivo General
~\\Diseñar y validar un plan de muestreo para aceptación y rechazo de lotes de cítricos en viveros del Valle del Cauca que permita estimar la cantidad de plantas infectadas con el virus de la tristeza en el lote.
\item Objetivos Específicos
\begin{itemize}
\item[-]Proponer y diseñar diferentes tipos de muestreo tipo aceptación/rechazo para lotes de cítricos en viveros del Valle del Cauca.
\item[-]Validar los diseños muéstrales por medio de simulación teniendo en cuenta confianza y costo del muestreo.
\item[-]Estimar la cantidad de plantas infectadas con el virus de la tristeza.
\end{itemize}
\end{itemize}
\end{frame}

\begin{frame}
\frametitle{Variable de interés}
~\\Nuestra variable de interés se puede definir de la siguiente manera:
$$X= \left\{\begin{array}{cc}
             1 &   si \; la \; planta \; esta \; infectada \\
             \\ 0 &  si \; la \; planta \; no \; esta \; infectada \\
             \end{array}
   \right.$$
\end{frame}


\begin{frame}
\frametitle{Variable de interés}
~\\Para llegar a la variable de interés definida anteriormente, se lleva a cabo una prueba de laboratorio en la cuál se pueden medir hasta 45 muestras de tejido de plantas, donde se obtiene al final una coloración en el recipiente, a esta coloración se le hace una lectura visual o colorimétrica, usualmente se utiliza más la medida colorimétrica por ser mas objetiva, esta se obtiene con un valor de absorbancia, que corresponde en pocas palabras a una cuantificación de la percepción del color. Se encuentra de la siguiente manera:
\[ A_\lambda=-log_{10} \left( \frac{I}{I_0}\right) \]
~\\Donde:
~\\I es la intensidad de la luz con una longitud de onda específica $\lambda$ tras haber atravesado una muestra (intensidad de la luz transmitida).
~\\$I_0$  es la intensidad de la luz antes de entrar a la muestra (intensidad de la luz incidente).
\end{frame}

\begin{frame}
\frametitle{Antecedentes}
\begin{itemize}
\item Antecedentes Contextuales
\begin{itemize}
\item[1.]Vidal, E. (2010). Epidemiología de Plum pox virus y citrus tristeza virus en bloques de plantas de vivero. Métodos de control (Tesis doctoral). Universitat Politécnica de Valencia, Valencia, España.
\item[2.]Pérez,A. , Mora, B. , León, M. , García, J. \& Abad, P. (2012). Enfermedades causadas por Phytophthora en viveros de plantas ornamentales (Artículo). Universitat Politécnica de Valencia, Valencia, España. Bol. San. Veg. Plagas, 38: 143-156.
\item[3.]Parke, J. L., Knaus, B. J., Fieland, V. J., Lewis, C., and Grünwald, N. J.
2014. Phytophthora community structure analyses in Oregon nurseries
inform systems approaches to disease management. Phytopathology
104:1052-1062.
\end{itemize}
\end{itemize}
\end{frame}

\begin{frame}
\frametitle{Antecedentes}
\begin{itemize}
\item Antecedentes Contextuales
\begin{itemize}
\item[4.]Caribbean food crops society. (2008).El Virus de la Tristeza de los Citricos (CTV) en Plantaciones Comerciales y Viveros
de la Repûblica Dominicana.(Plenary Session and Oral Presentations). University of Florida.
\item[5.]Melgara,(2018).Ocurrencia de Huanglongbing (Candidatus Liberibacter asiaticus) y su vector [Diaphorina citri Kuwayama (Hemiptera: Liviidae)] en viveros de cítricos de Masaya.(Trabajo de Graduación, MAESTRÍA EN SANIDAD VEGETAL).UNIVERSIDAD NACIONAL AGRARIA
FACULTAD DE AGRONOMIA
\end{itemize}
\end{itemize}
\end{frame}

\begin{frame}
\frametitle{Antecedentes}
\begin{itemize}
\item Antecedentes Estadísticos
\begin{itemize}
\item[1.]Abad Corpa, E. , Leal Llopis, J. , Paredes Sidrach de Cardona, A. \& García Palomares, A. (2004). Monitorización del cumplimiento del protocolo de mantenimiento de la cateterización venosa mediante el método LQAS (Artículo). Universidad de Murcia, España. Revista electrónica semestral de enfermería.
\item[2.] Tawfik, Y. , Hoque, S. \&  Siddiqi, M. (2001). Using lot quality assurance sampling to improve immunization coverage in Bangladesh (Artículo). Bulletin of the world health organization, 79.
\end{itemize}
\end{itemize}
\end{frame}

\begin{frame}
\frametitle{Antecedentes}
\begin{itemize}
\item Antecedentes Estadísticos
\begin{itemize}
\item[3.]José A Andreo, Matilde Barrio, Rosa M Ramos, Miguel Torralba, Faustino Herrero \& Pedro J Saturno. (2000). Evaluación, mejora y monitorización de la adecuación de ingreso y estancia en Medicina Interna con el muestreo de aceptación de lotes (Artículo). Instituto nación de salud pública, México. ResearchGate.
\item[4.]Mark Myatt, Hans Limburg, Darwin Minassian \& Damson Katyola. (2003). Field trial of applicability of lot quality assurance sampling survey method for rapid assessment of prevalence of active tracoma. (Artículo). Bulletin of the world health organization, 81.
\end{itemize}
\end{itemize}
\end{frame}

\begin{frame}
\frametitle{Marco Teórico}
~\\El marco teórico está compuesto por algunas definiciones y teorías tanto contextuales como estadísticas necesarias en la investigación.
~\\\textbf{Marco conceptual:}
\begin{itemize}
\item Vivero: Toda porción de terreno o medio de cultivo dedicado a la multiplicación de plantas, a su crianza o a su conservación.
\item Lote: Conjunto de unidades de un solo producto básico, identificable por su composición homogénea, origen, etc., que forma parte de un envío [FAO, 1990] 
\item Plaga: Cualquier especie, raza o biotipo vegetal o animal o agente patógeno dañino para las plantas o productos vegetales [FAO 1990; revisado FAO, 1995; CIPF, 1997] 
\item Prueba: Examen oficial, no visual, para determinar la presencia de plagas o para identificar tales plagas [FAO, 1990] 
\end{itemize}
\end{frame}

\begin{frame}
\frametitle{Marco Teórico}
\begin{itemize}
\item Virus de la tristeza: El virus de la tristeza de los cítricos (Citrus tristeza virus, CTV) causa una de las enfermedades más
dañinas de este cultivo.). Se refiere al decaimiento observado en muchas especies de cítricos injertados sobre patrones de Citrus aurantium (naranjo amargo) o de Citrus limon (limonero); algunas cepas del CTV inducen otros síndromes, como acanaladuras o picado del tallo, enanismo, menor productividad y baja calidad del fruto en muchos cultivares comerciales,incluso en ejemplares injertados sobre patrones tolerantes a la tristeza.
\item Huanglongbing(HLB):Es una de las enfermedades más peligrosas y temidas por las pérdidas productivas y económicas que ocasiona. Las plantas jóvenes afectadas no entran nunca en producción y las plantas adultas dejarán de producir pocos años después de que se manifiesta la enfermedad. En las plantas de vivero infectadas, los síntomas pueden ser esporádicos e inconsistentes aunque un porcentaje alto de plantas se encuentren contaminadas.
\end{itemize}
\end{frame}

\begin{frame}
\frametitle{Marco Teórico}
\begin{itemize}
\item Leprosis: Enfermedad viral que se trasmite por ácaros del genero Brevipalpus spp. La leprosis es causada por un virus, que es transmitido por un ácaro o arañuela, es una enfermedad en los naranjos, mandarinas y otros cítricos. Primero salen manchas amarillas en las hojas y frutos. En los tallos las manchas son de color café con grietas, el árbol va muriendo gradualmente y el daño más importante es la caída prematura de los frutos, a su vez las manchas en los frutos bajan el valor de los mismos.
\item Exocortis: Es una enfermedad producida por el viroide de la exocortis de los cítricos (CEVd), un agente patógeno mucho más pequeño que los virus. Se caracteriza por la aparición de escamas
y grietas verticales en la corteza, manchas amarillas en los
brotes tiernos y enanismo, en especies sensibles.
\end{itemize}
\end{frame}

\begin{frame}
\frametitle{Marco Teórico}
~\\\textbf{Marco Estadístico:}
\begin{itemize}
\item Muestreo: Es el proceso mediante el cuál se extrae un conjunto de unidades ó individuos de una población con el objetivo de analizarlos e intentar caracterizar el total de la población. Existen dos tipos de muestreo desde el punto de vista estadístico:
\item[-] Muestreo probabilístico: Todos los elementos de la población deben tener la misma probabilidad de ser seleccionados. Dentro de este tipo de muestreo tenemos los siguientes:
\begin{itemize}
\item Muestreo Aleatorio Simple (MAS):Se trata de un procedimiento de selección con probabilidades iguales que consiste en obtener la muestra unidad a unidad de forma aleatoria.

\end{itemize}
\end{itemize}
\end{frame}

\begin{frame}
\frametitle{Marco Teórico}
\begin{itemize}
~\\
\begin{itemize}
\item Muestreo sistemático: El muestreo sistemático consiste en retirar una muestra de las unidades del lote a intervalos fijos y predeterminados. Sin
embargo, la primera selección debe hacerse al azar en el lote. Dos ventajas de este método son que una maquinaria podrá automatizar el proceso de muestreo y que sólo se requiere utilizar un proceso aleatorio para seleccionar la primera unidad.
\item Muestreo estratificado: El muestreo estratificado consiste en separar el lote en subdivisiones distintas (es decir, en estratos) para luego extraer
unidades de muestra de todas y cada una de las subdivisiones. Dentro de cada subdivisión, las unidades de muestra se
retiran utilizando un método particular (sistemático o aleatorio). En ciertos casos, se podrán tomar distintos números de
unidades muestrales de cada subdivisión; por ejemplo, el número de muestras podrá ser proporcional al tamaño de la
subdivisión o podrá basarse en conocimiento previo sobre la infestación de las subdivisiones.
\end{itemize}
\end{itemize}
\end{frame}

\begin{frame}
\frametitle{Marco Teórico}
\begin{itemize}
~\\
\begin{itemize}
\item Muestreo secuencial: El muestreo secuencial consiste en retirar una serie de unidades de muestra utilizando uno de los métodos anteriores. Después de retirar cada muestra (o grupo), se acumulan los datos y se comparan con rangos predeterminados, para decidir si se aceptará o rechazará el lote, o si se continuará con el muestreo.
\item Muestreo por conglomerados: Es una técnica utilizada cuando hay agrupamientos "naturales" relativamente homogéneos en una población estadística. En esta técnica, la población total se divide en estos grupos (o clusters) y una muestra aleatoria simple se selecciona de los grupos.
\end{itemize}
\end{itemize}
\end{frame}

\begin{frame}
\frametitle{Marco Teórico}
\begin{itemize}
\item[-] Muestreo no probabilístico: No se conoce la probabilidad que tienen los diferentes elementos de la población de estudio de ser seleccionados.
\begin{itemize}
\item Muestreo de conveniencia: El muestreo de conveniencia consiste en seleccionar las unidades más convenientes (por ejemplo, las más accesibles,
económicas, rápidas) del lote, sin seleccionar las unidades en forma aleatoria o sistemática.
\item Muestreo arbitrario: El muestreo arbitrario consiste en seleccionar unidades arbitrarias sin utilizar un verdadero proceso de aleatoriedad, lo
cual suele parecer aleatorio debido a que el inspector no está consciente de ningún sesgo en la selección. Sin embargo,
puede existir un sesgo inconsciente, de modo que se desconoce en qué medida la muestra es representativa del lote.
\end{itemize}
\end{itemize}
\end{frame}

\begin{frame}
\frametitle{Marco Teórico}
\begin{itemize}
~\\
\begin{itemize}
\item Muestreo selectivo o dirigido: El muestreo selectivo consiste en seleccionar deliberadamente muestras de las partes del lote que más probabilidad
tienen de estar infestadas o en seleccionar unidades que están obviamente infestadas, para aumentar la probabilidad de
detectar una plaga reglamentada específica. Este método podrá depender de inspectores que tengan experiencia con el
producto y que conozcan bien la biología de la plaga.
\end{itemize}
\end{itemize}
\end{frame}

\begin{frame}
\frametitle{Marco Teórico}
\begin{itemize}
\item Muestreo de aceptación: Es un procedimiento para obtener información de un proceso (población) con base en cierto número de observaciones (muestra), con la finalidad de inferir la calidad de dicho proceso. Escalante (2006).
~\\El procedimiento estadístico del muestreo de aceptación se basa en la metodología de la prueba de hipótesis. Las hipótesis nula y alternativa son las siguientes:
$$H_0:La \; calidad \; del \; lote \; es \; buena$$
$$H_a:La \; calidad \; del \; lote \; es \; mala$$
\end{itemize}
\end{frame}

\begin{frame}
\frametitle{Marco Teórico}
\begin{figure}[!h]
        \includegraphics[width=9.5cm]{IMAGENES/MA.png}
        \label{figura1}
\end{figure}
\end{frame}

\begin{frame}
\frametitle{Marco Teórico}
~\\El muestreo de aceptación puede dividirse en dos tipos fundamentales dependiendo de la característica observada:
\begin{itemize}
\item Muestreo por atributos: cuando en la inspección los artículos se dividen en defectuosos y en no defectuosos, según cumplan con un conjunto de requerimientos.
\item Muestreo por variables: cuando en la inspección se mide una variable cuantitativa: longitudes, pesos... y se evalúa la distancia entre dicha cantidad y la requerida en las especificaciones.
\end{itemize}
~\\En el muestreo de aceptación se utilizan principalmente tres distribuciones de probabilidad dependiendo del tamaño del lote(grande o pequeño), las distribuciones utilizadas son la Hipergeométrica, la Poisson y la Binomial.
\end{frame}

\begin{frame}
\frametitle{Marco Teórico}
\begin{itemize}
\item Distribución hipergeométrica:La distribución hipergeométrica es fundamental para gran parte del muestreo de aceptación. Es aplicable cuando se muestrea una característica de atributo de un lote finito ó pequeño sin reemplazo. Su función de probabilidad es:
$$f(x)=\frac{\binom{N_p}{x}\binom{N_q}{n-x}}{\binom{N}{n}}$$
~\\ Donde; 
~\\ N es el tamaño del lote, $N>0$
~\\ p es la proporción defectuosa en el lote, $p=0, 1/N, 2/N, \cdots , 1$
~\\ q es la proporción efectiva en el lote, $q = 1 – p$
~\\ n es el tamaño de la muestra, $n = 1, 2, …, N$
~\\ x es el número de ocurrencias, $x = 0, 1, 2, …, n$
\end{itemize}
\end{frame}

\begin{frame}
\frametitle{Marco Teórico}
\begin{itemize}
\item Distribución binomial: Es la distribución más utilizada en el muestreo de aceptación. Complementa la hipergeométrica en el sentido de que se emplea al muestrear una característica de atributo de un lote (o proceso) infinito ó grande, o un lote finito cuando se toma una muestra con reemplazo. Su función de probabilidad es:
$$f(x)=\binom{n}{x} \; p^x \; (1-p)^{n-x}=\binom{n}{x} \; p^x \; q^{n-x}$$
~\\ Donde; 
~\\ n es el tamaño de la muestra, $n>0$
~\\ p es la proporción defectuosa, $0\leq p \leq 1$
~\\ q es la proporción efectiva, $q = 1 – p$
~\\ x es el número de ocurrencias, $x = 0, 1, 2, …, n$
\end{itemize}
\end{frame}

\begin{frame}
\frametitle{Marco Teórico}
\begin{itemize}
\item Distribución poisson: La distribución de Poisson se utiliza para calcular las características de los planes de muestreo, que especifican un número dado de defectos por unidad, como el número de remaches defectuosos en el ala de un avión o el número de piedras permitido en un pedazo de vidrio de un tamaño determinado. El parámetro en la distribución de Poisson es simplemente $\mu$. Su función de probabilidad es:
$$f(x)=\frac{{\mu}^x \; e^{-\mu}}{x!}$$
~\\ Donde; 
~\\ $\mu$ es el número medio de defectos, $\mu>0$
~\\ x es el número de ocurrencias, $x=0,1,2,\cdots$
\end{itemize}
\end{frame}

\begin{frame}
\frametitle{Metodología}
\end{frame}
\end{document}
\documentclass[12pt]{beamer}
\usetheme{CambridgeUS}
\usepackage[utf8]{inputenc}
\usepackage[spanish]{babel}
\usepackage{amsmath}
\usepackage{amsfonts}
\usepackage{amssymb}
\usepackage{graphicx}
\usepackage{ragged2e}
\setbeamertemplate{navigation symbols}{} 
\author[Kevin García - Alejandro Vargas]{Kevin García 1533173 \newline Alejandro Vargas 1525953}
\title[Anteproyecto]{Diseño y validación de muestreo de cítricos para detección del virus de la tristeza en viveros del Valle del Cauca}


\newcommand\Wider[2][3em]{%
\makebox[\linewidth][c]{%
  \begin{minipage}{\dimexpr\textwidth+#1\relax}
  \raggedright#2
  \end{minipage}%
  }%
}


%\setbeamercovered{transparent} 
%\setbeamertemplate{navigation symbols}{} 
%\logo{} 
%\institute{} 
%\date{} 
%\subject{} 
\begin{document}
\justify
\begin{frame}
\titlepage
\end{frame}

%\begin{frame}
%\tableofcontents
%\end{frame}
\begin{frame}
\frametitle{Información del proyecto}
\begin{itemize}
\item Entidad encargada: AGROSAVIA (Corpoica)
~\\La Corporación Colombiana de Investigación Agropecuaria, Corpoica, es una entidad pública descentralizada de participación mixta sin ánimo de lucro, de carácter científico y técnico, cuyo objeto es desarrollar y ejecutar actividades de Investigación, Tecnología y transferir procesos de Innovación tecnológica al sector agropecuario.

~\\
\item Personal a cargo:
\begin{itemize}
\item[-]Nubia Murcia Riaño (Investigador Ph.D.)
\item[-]Mauricio Fernando Martínez (Investigador Máster)
\item[-]Elizabeth Narvaez Toro (Líder de Seguimiento y Evaluación)
\end{itemize}
\end{itemize}
\end{frame}


\begin{frame}
\frametitle{Problema Contextual}
~\\Existen varias especies de áfidos(pulgones) que infectan los cítricos y generan una enfermedad llamada ``virus de la tristeza'', el cuál causa una de las enfermedades más dañinas de este cultivo (debilita el árbol, que da producciones escasas, llegando finalmente a matarlo en varios meses o si el virus es violento, dos o tres semanas). El inconveniente es que esta enfermedad es asintomática, es decir, no podemos diferenciar a simple vista una planta infectada con una no infectada. Al sembrar una planta con esta infección desde el comienzo, se perderá mucho dinero invirtiendo en su mantenimiento, por lo cuál se necesita asegurar o garantizar que las plantas que van a ser sembradas y entregadas estén limpias de esta enfermedad, logrando de esta manera la producción de material certificado.
\end{frame}

\begin{frame}
\frametitle{Problema Estadístico}
~\\Diseñar, plantear y validar métodos de muestreo tipo aceptación y rechazo, confiables y económicos, en lotes de cítricos criados en viveros del Valle del Cauca con el fin de detectar el virus de la tristeza en edades tempranas de la planta, creando una norma que ayude a detectar a tiempo los lotes infectados evitando perdidas económicas futuras.
\end{frame}

\begin{frame}
\frametitle{Objetivos propuestos}
\begin{itemize}
\item Objetivo General
~\\Diseñar y validar un muestreo de cítricos en viveros del Valle del Cauca que permita estimar la cantidad de plantas infectadas con el virus de la tristeza en un lote de plantas.
\item Objetivos Específicos
\begin{itemize}
\item[-]Proponer y diseñar diferentes tipos de muestreo tipo aceptación/rechazo de lotes para cítricos en viveros del Valle del Cauca.
\item[-]Validar los diseños muéstrales por medio de simulación teniendo en cuenta confianza y costo del muestreo.
\item[-]Estimar la cantidad de plantas infectadas con el virus de la tristeza.
\item[-]Plantear normas y metodologías a seguir para la obtención de la muestra.
\end{itemize}
\end{itemize}
\end{frame}
\end{document}
\chapter{Planteamiento de la investigaci\'{o}n}
Existen diversas enfermedades en los c\'{i}tricos transmitidas principalmente por injertaci\'{o}n, vectores (organismos o insectos), y uso de herramienta, las cuales son muy da\~{n}inas para este cultivo, entre ellas, el virus de la tristeza, HLB, Leprosis y Exocortis. Estas debilitan el \'{a}rbol, generando producciones escasas, y en casos avanzados puede llegar a matar el \'{a}rbol. El inconveniente es que estas enfermedades son asintom\'{a}ticas en edades tempranas de la planta, es decir, no podemos diferenciar a simple vista una planta infectada con una no infectada. Al sembrar una planta con alguna de estas infecciones desde el comienzo, se perder\'{a} mucho dinero invirtiendo en su mantenimiento, por lo cu\'{a}l se necesita asegurar o garantizar que las plantas que van a ser sembradas y entregadas est\'{e}n limpias de estas enfermedades, logrando de esta manera la producci\'{o}n de material certificado.\\

\section{Problema Estad\'{i}stico}
Dada la imposibilidad de tener s\'{i}ntomas o alg\'{u}n efecto visible con el cual se pueda aumentar la probabilidad de que la planta seleccionada este infectada...
%En la pr\'{a}ctica, muchas de las variables aleatorias que se estudian, se mueven en espacios acotados, de all\'{i} que encontrar distribuciones probabil\'{i}sticas que modelen de forma adecuada dichos fen\'{o}menos cobra vital importancia. Una alternativa para lo anterior es la distribuci\'{o}n beta, pues seg\'{u}n Canavos (1998) `` Se ha utilizado para representar variables f\'{i}sicas cuyos valores se encuentran restringidos a un intervalo de longitud finita y para encontrar ciertas cantidades que se encuentran como l\'{i}mites de tolerancia sin necesidad de la hip\'{o}tesis de una distribuci\'{o}n normal". Tambi\'{e}n, Jhonson Norman dice que ...\\

%Espec\'{i}ficamente, cuando se trata de modelar proporciones, indicadores de desarrollo, tasas y otros fen\'{o}menos cuyos rangos de movimiento est\'{a}n definidos en el espacio [0, 1], se emplea el modelo probabil\'{i}stico beta est\'{a}ndar, el cual es un caso especial del modelo beta generalizado, y es el que se tratar\'{a} en el presente trabajo de grado. Actualmente, la mayor\'{i}a de los autores que trabajan con esta distribuci\'{o}n, se apoyan en la investigaci\'{o}n y resultados que se han obtenido en ella desde el paradigma de la estad\'{i}stica cl\'{a}sica. Por ende, al realizar estimaciones puntuales de los par\'{i}metros de forma, recurren a los ya conocidos m\'{e}todos de los momentos y de m\'{a}xima verosimilitud. En este \'{u}ltimo, se apoyan en m\'{e}todos iterativos como Fisher Scoring o Newton Rapson, pues a la hora de derivar la funci\'{o}n log-verosimilitud con respecto a los par\'{a}metros, la expresi\'{o}n se indetermina.\\

%En consecuencia, si se presentan problemas tales como tama\~{n}os mu\'{e}strales muy peque\~{n}os o que la muestra que se tenga a disposici\'{o}n no este aleatorizada, se tendr\'{a}n inconvenientes a la hora de sacar conclusiones en la investigaci\'{o}n, pues estas carecer\'{a}n de validez externa. ``Este tipo de validez, est\'{a} relacionada con la selecci\'{o}n de las unidades estad\'{i}sticas que ser\'{a}n medidas o encuestadas, aqu\'{i} la muestra debe ser representativa de la poblaci\'{o}n, de la selecci\'{o}n dada y para ello se necesita que la forma como ella sea seleccionada no tenga sesgos, lo cual se garantiza mediante el uso del muestreo probabil\'{i}stico", Klinger (2011). En efecto, si se tienen dificultades en el paradigma cl\'{a}sico o simplemente se desea contar con una fuente de estudio adicional para concluir, el paradigma bayesiano surge como una alternativa importante.\\

%Tener estimaciones bayesianas para cualquier caso de investigaci\'{o}n es beneficioso, pues esta enriquece el an\'{a}lisis, dado que los estimadores que se obtienen consideran el conocimiento o experiencia subjetiva que tienen los expertos en el problema objeto de estudio, para Gelman (2014) ``Una raz\'{o}n pragm\'{a}tica para el uso de los m\'{e}todos bayesianos es la flexibilidad inherente introducida por su incorporaci\'{o}n de m\'{u}ltiples niveles de aleatoriedad y la capacidad resultante de combinar informaci\'{o}n de diferentes fuentes, al tiempo que incorpora todas las fuentes razonables de incertidumbre en los res\'{u}menes inferenciales. Tales m\'{e}todos conducen naturalmente a estimaciones suavizadas en estructuras de datos complicadas y, por consiguiente, tienen la capacidad de obtener mejores respuestas del mundo real". Sin embargo, si se quiere ahondar en la inferencia bayesiana para esta distribuci\'{o}n, se encuentran muchas limitaciones, ya que hasta el momento en la literatura no se cuenta con una distribuci\'{o}n a priori para los par\'{a}metros de la beta est\'{a}ndar, entre otras causas porque no se les ha dado interpretaci\'{o}n alguna.\\

\subsection{Justificaci\'{o}n}
Nuestro rol como estad\'{i}sticos es de vital importancia para lograr verificar que la producci\'{o}n est\'{e} libre de cualquier plaga y mitigar en gran medida posibles p\'{e}rdidas en toda la industria por lotes infectados, logrando que productores y consumidores se ven beneficiados.

%En la literatura se encuentran muchos trabajos en los que se utiliza la Distribuci\'{o}n Beta, pero en estos no se tienen trabajos en los que se estimen los par\'{a}metros de esta distribuci\'{o}n de forma bayesiana. Y en la pr\'{a}ctica, muchas de las variables se mueven en el espacio $\left[0,1\right] $ \'{o} en espacios acotados, y una buena distribuci\'{o}n para modelar estas variables es la Distribuci\'{o}n Beta. Pero no se puede tener una buena estimaci\'{o}n ya que la forma anal\'{i}tica de la distribuci\'{o}n es muy compleja y se hace dif\'{i}cil utilizar la distribuci\'{o}n para asociarla a una verosimilitud y poder obtener resultados.\\

%Dado los problemas que se tienen en el enfoque cl\'{a}sico  a la hora de estimar los par\'{a}metros de la distribuci\'{o}n Beta, los cuales no permiten realizar inferencias v\'{a}lidas. Se hace necesario proponer estimaciones para dichos par\'{a}metros desde el enfoque bayesiano, obteniendo as\'{i} conclusiones que permitan superar las anteriores dificultades, y as\'{i} el estudio estad\'{i}stico tenga validez.`` Es b\'{a}sico que la realidad investigada quede representada correctamente en la estrategia de la investigaci\'{o}n utilizada, o de lo contrario este proceso carecer\'{a} de validez"  Klinger (2011)\\



\subsection{Objetivos}
\subsubsection{Objetivo General}
\begin{itemize}
\item Dise\~{n}ar y validar un plan de muestreo para aceptaci\'{o}n y rechazo de lotes de c\'{i}tricos en viveros del Valle del Cauca que permita estimar la cantidad de plantas infectadas en el lote..
\end{itemize}
\subsubsection{Objetivos Espec\'{i}ficos}

\begin{itemize}
\item Proponer y dise\~{n}ar diferentes tipos de muestreo tipo aceptaci\'{o}n/rechazo para lotes de c\'{i}tricos en viveros del Valle del Cauca.
%\item Proponer una distribuci\'{o}n a priori para la media y la varianza de la Distribuci\'{o}n Beta. 
\item Validar los dise\~{n}os mu\'{e}strales por medio de simulaci\'{o}n teniendo en cuenta confianza y costo del muestreo.
%\item Proponer una distribuci\'{o}n a priori conjunta para la media y la varianza de la Distribuci\'{o}n Beta.
\item Estimar la cantidad de plantas infectadas en el lote.
%\item Proponer una distribuci\'{o}n a priori para los par\'{a}metros de la Distribucion Beta, a trav\'{e}s de la distribuci\'{o}n conjunta planteada para su media y varianza.

\end{itemize}


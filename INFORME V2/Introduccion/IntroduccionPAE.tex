\chapter{Introducci\'{o}n}

\section{Planteamiento del problema}
~\\En Colombia existen 97.275 hect\'{a}reas sembradas de c\'{i}tricos entre cultivos de naranja, lim\'{o}n, mandarina, toronja, tangelo, pomelo y lima, seg\'{u}n datos del Ministerio de Agricultura y Desarrollo Rural, es el grupo de frutales con mayor \'{a}rea sembrada en el pa\'{i}s despu\'{e}s del pl\'{a}tano, y genera aproximadamente 413.374 empleos directos e indirectos.

~\\Seg\'{u}n la organizaci\'{o}n de las Naciones Unidas para la Alimentaci\'{o}n y la Agricultura (FAO), el territorio colombiano es una de las siete naciones que puede volverse despensa mundial de alimentos, gracias a que tiene suficiente tierra para ampliar la frontera agr\'{i}cola sin necesidad de talar bosques. Por otro lado, \'{e}ste goza de privilegios naturales como ser el tercer pa\'{i}s con mayores recursos de agua y con diversidad clim\'{a}tica. A pesar de las ventajas comparativas que ofrecen muchas regiones del pa\'{i}s para el desarrollo citr\'{i}cola, la falta de escalas comerciales significativas, la alta dispersi\'{o}n geogr\'{a}fica de la producci\'{o}n, la falta de gesti\'{o}n empresarial y de desarrollo tecnol\'{o}gico, hacen que la producci\'{o}n y comercializaci\'{o}n de c\'{i}tricos sean poco competitivos en el mercado nacional e internacional. Adem\'{a}s, el pa\'{i}s enfrenta problemas para incursionar en los mercados externos debido a que, entre otros factores, no se cuenta con las variedades ni calidades adecuadas requeridas, no hay continuidad en la oferta exportable e igualmente se deben superar problemas de empaque y presentaciones, as\'{i} como barreras t\'{e}cnicas y sanitarias. Inclusive, existe poco grado de integraci\'{o}n entre la industria y la agricultura, no hay material vegetal certificado, falta investigaci\'{o}n y transferencia de tecnolog\'{i}a (desarrollo de variedades y calidades) en la fase agr\'{i}cola y agroindustrial, as\'{i} como prevenci\'{o}n de plagas y enfermedades.

~\\Existen diversas enfermedades que afectan a los c\'{i}tricos transmitidas principalmente por injertaci\'{o}n, vectores (organismos o insectos), y uso de herramienta, las cuales son muy da\~{n}inas para este cultivo. Las enfermedades que se presentan con mayor frecuencia en Colombia y las m\'{a}s importantes son el virus de la tristeza, Huanglongbing(HLB), Leprosis y Exocortis; cada una de ellas posee caracter\'{i}sticas espec\'{i}ficas en cuanto a su sintomatolog\'{i}a y consecuencias, \'{e}stas debilitan el \'{a}rbol, generando producciones escasas o con un valor inferior al establecido, y en casos avanzados pueden llegar a matar el \'{a}rbol. Sin embargo, en el pa\'{i}s no se ha implementado o desarrollado un sistema de certificaci\'{o}n de material vegetal que garantice la calidad de la propagaci\'{o}n y la seguridad de la especie.

~\\El problema principal es que la mayor\'{i}a de estas enfermedades son asintom\'{a}ticas en la etapa de vivero (edades tempranas de la planta) que tiene una duraci\'{o}n de 12 a 36 meses, es decir, en esta etapa no se puede diferenciar a simple vista una planta infectada con una no infectada, por lo que se hace necesario aplicar una prueba serol\'{o}gica para saber el verdadero estado de la planta. Al sembrar una planta con alguna de estas infecciones desde el comienzo, se perder\'{i}a mucho dinero invirtiendo en su mantenimiento y no se obtendr\'{i}an las ganancias o productos esperados, por lo cual se necesita asegurar o garantizar que las plantas que van a ser sembradas y entregadas est\'{e}n limpias de \'{e}stas enfermedades, logrando de esta manera la producci\'{o}n de material certificado. Dado que para evaluar las plantas se debe realizar la prueba serol\'{o}gica DAS-ELISA, y los lotes de c\'{i}tricos por lo general tienen una poblaci\'{o}n considerablemente grande, es imposible realizar un censo a todos los lotes que van a ser entregados por log\'{i}stica y econom\'{i}a. Por lo que a partir de esto surge la pregunta: ?`Es posible dise\~{n}ar un plan de muestreo en viveros, que permita la detecci\'{o}n temprana de \'{e}stas enfermedades en los c\'{i}tricos?


\section{Justificaci\'{o}n}
~\\Como se mencion\'{o} anteriormente, los frutales de c\'{i}tricos son las plantas m\'{a}s sembradas en Colombia despu\'{e}s del pl\'{a}tano generando a su vez miles de empleos directos e indirectos, siendo un bien sumamente importante para la econom\'{i}a del pa\'{i}s y en el caso en que este se vea afectado, as\'{i} mismo se ver\'{a} afectado todo el sector y la econom\'{i}a misma. 

~\\Con el prop\'{o}sito de evitar epidemias en toda la poblaci\'{o}n de c\'{i}tricos, nace la necesidad de regular la forma en que los viveros producen dichas plantas y certificar los lotes con el fin de que no se reproduzcan plantas infectadas, debido a que estas enfermedades como lo son el virus de la tristeza, HLB, Leprosis, Exocortis, entre otras, pueden ocasionar una disminuci\'{o}n considerable en la producci\'{o}n, es decir, pueden generarse p\'{e}rdidas de lotes enteros o que la calidad de las frutas est\'{e} por debajo de lo esperado. 

~\\En particular, desde el \'{a}rea de la estad\'{i}stica se han realizado muchos estudios con lotes de c\'{i}tricos criados en viveros, pero estos se orientan a evaluar efectos que tienen ciertos tratamientos sobre las plantas, es decir, evaluar rendimiento, producci\'{o}n, crecimiento, entre otros factores, sin embargo, dichos estudios no se han centrado en validar si las muestras que toman son representativas del lote en general y si dichas muestras permiten verificar si las plantas presentan o no enfermedades. Adem\'{a}s, la implementaci\'{o}n de un buen plan de muestreo en estos viveros definir\'{a} si un lote puede llegar a catalogarse como ``bueno"  o si por el contrario hay que descartarlo como un ``lote infectado".

~\\El hecho de poder discernir entre qu\'{e} lotes saldr\'{a}n a la venta y cuales no deber\'{i}an, evitar\'{a} a largo plazo posibles plagas de estas enfermedades, las cuales actualmente ya se est\'{a}n viendo propagadas en varias regiones del pa\'{i}s. Tambi\'{e}n es relevante conocer qu\'{e} tipo de plantas se est\'{a}n entregando a los productores y finalmente cu\'{a}l es la calidad de cultivo de c\'{i}tricos que tenemos en nuestra regi\'{o}n, esto permitir\'{a} no solo ser competentes sino tambi\'{e}n sostener una econom\'{i}a que gira alrededor de estos productos agr\'{i}colas.

~\\Cabe resaltar que los planes de muestreo que finalmente se desarrollar\'{a}n pueden llegar a ser de utilidad no solo para el sector de los c\'{i}tricos, por lo que podr\'{i}a ampliarse y ser de utilidad para detectar enfermedades en plantas de vivero que compartan caracter\'{i}sticas similares, lo cual hace que esta labor no sea solo un aporte para un sector en particular si no para la agricultura en general. 


\section{Objetivos}
\subsection{Objetivo General}
\begin{itemize}
\item Dise\~{n}ar y validar emp\'{i}ricamente un plan de muestreo para aceptaci\'{o}n y rechazo de lotes de c\'{i}tricos en viveros del Valle del Cauca que permita estimar la cantidad de plantas infectadas en el lote.
\end{itemize}
\subsection{Objetivos Espec\'{i}ficos}
\begin{itemize}
\item Proponer y dise\~{n}ar diferentes tipos de muestreo tipo aceptaci\'{o}n/rechazo para lotes de c\'{i}tricos en viveros del Valle del Cauca.
\item Validar los dise\~{n}os mu\'{e}strales por medio de estudios de simulaci\'{o}n.
\item Estimar la cantidad de plantas infectadas en el lote.
\end{itemize}
\section{Antecedentes}
~\\A continuaci\'{o}n se muestran algunos antecedentes tanto contextuales como estad\'{i}sticos. El primer grupo de investigaciones, son estudios sobre an\'{a}lisis de estas enfermedades en lotes de plantas, a pesar de que no tienen un an\'{a}lisis estad\'{i}stico, han sido de gran ayuda, ya que se aplicaron diferentes tipos de muestreo para la evaluaci\'{o}n de estas enfermedades, y nos permiten tener una idea de la distribuci\'{o}n de las plantas en los lotes y porcentajes de infecci\'{o}n para realizar nuestras correspondientes simulaciones. El segundo grupo de investigaciones, corresponde a estudios fuera de nuestro contexto de inter\'{e}s, en los cuales se aplic\'{o} el muestreo de aceptaci\'{o}n y rechazo en distintas problem\'{a}ticas, a pesar de que todos estos estudios fueron en el \'{a}rea de la salud, son muy \'{u}tiles ya que nos centramos en analizar el funcionamiento de esta herramienta estad\'{i}stica para posteriormente llevarla a nuestro contexto.

\subsection{Antecedentes contextuales}
~\\\textbf{\citet{AC1}}
~\\El objetivo de esta tesis fue el estudio de los distintos factores que determinan la epidemiolog\'{i}a de PPV(plum pox virus) y CTV(Citrus tristeza virus) en vivero, con el fin de establecer posibles estrategias de control. 
~\\Para el primer virus (PPV) se dispuso de dos parcelas , una con alta incidencia y otra con baja incidencia del virus. La primera parcela(alta incidencia) se dividi\'{o} en dos bloques estad\'{i}sticos imaginarios o subparcelas, cada uno de los cuales estuvo formado por dos bloques de plantas. La subparcela 1 contaba con dos filas de plantas divididas en 4 grupos cada una, donde cada grupo contaba con un total de 40 plantas, por tanto se tuvo al final 8 grupos para un total de 320 plantas. A esta subparcela se le muestreo \'{u}nicamente una de las filas. En la subparcela 2 se plantaron 6 grupos de plantas, donde cada grupo consisti\'{o} en 2 filas de plantas, y cada fila se dividi\'{o} a su vez en 4 grupos, teniendo un total de 1040 plantas. En esta subparcela el muestreo se realiz\'{o} tomando las 4 primeras filas. La segunda parcela(baja incidencia) tambi\'{e}n se dividi\'{o} en 2 subparcelas. La subparcela 1 estuvo constituida por 6 grupos de plantas, cada grupo formado por dos filas y cada fila formada por 6 bloques. El n\'{u}mero total de plantas por bloque fue de 45, para un total de plantas de 3240. Esta subparcela se muestreo tomando al azar 5 plantas de cada uno de los 72 bloques.
~\\Para el segundo virus (CTV) se dispuso de una parcela con alta incidencia. Esta tambi\'{e}n se dividi\'{o} en dos subparcelas. La subparcela 1 estuvo formada por 5 filas, donde cada fila conten\'{i}a 140 plantas para un total de 700 plantas. El muestreo para determinar la incidencia viral en la subparcela 1 se hizo tomando 2 hojas por planta.
~\\Los resultados arrojados fueron que no se detect\'{o} PPV en los an\'{a}lisis realizados en las dos parcelas(alta y baja incidencia) y tampoco se detect\'{o} la presencia de CTV en la parcela(alta incidencia).

~\\\textbf{\citet{AC2}}
~\\El objetivo de este trabajo fue estudiar las enfermedades causadas por las especies de Phytophthora. Se exploraron 23 viveros de plantas ornamentales, 19 en Valencia, 2 en Castell\'{o}n y 2 en Asturias muestreando solamente plantas sintom\'{a}ticas, analiz\'{a}ndose un total de 360 plantas pertenecientes a 56 g\'{e}neros diferentes. muestreando solamente plantas sintom\'{a}ticas, analizando un total de 360 plantas.
~\\En general se observ\'{o} una sintomatolog\'{i}a muy variada predominando los s\'{i}ntomas de seca parcial o total de la parte a\'{e}rea y se detect\'{o} la presencia de Phytophthora en 16 de los viveros analizados (70\%), en 12 de Valencia, 2 de Castell\'{o}n y 2 de Asturias.

~\\\textbf{\citet{AC3}}
~\\El objetivo fue describir la estructura de las comunidades de Phytophthora descubiertas durante un an\'{a}lisis de riesgo realizado en 4 viveros comerciales de Oreg\'{o}n durante un per\'{i}odo de 4 a\~{n}os. Se tomaron muestras de los 4 viveros cada 2 meses durante 4 a\~{n}os, recolectando 5 plantas de cada uno de los cuatro g\'{e}neros suceptibles a Phytophthora en cada fecha de muestreo. Se seleccionaron plantas sintom\'{a}ticas para maximizar la probabilidad de que Phytophthora ser\'{i}a detectado. Si no se encontraros plantas sintom\'{a}ticas, en su lugar se tomaron muestras de plantas asintom\'{a}ticas.
~\\De los 6811 cultivos aislados durante el estudio de 4 a\~{n}os, 1269(18.6\%) aislamientos fueron enviados para identificaci\'{o}n molecular y 674 de ellos fueron identificados como aislamientos de Phytophthora.

~\\\textbf{\citet{AC4}}
~\\En este caso se quer\'{i}a estudiar el estado del virus de la tristeza de los c\'{i}tricos en viveros y plantaciones comerciales,se tomaron muestras de 9 viveros de la rep\'{u}blica dominica y para cada uno de ellos se tomaron muestras del 1\% o 2\% del total de las plantas (1\% para lotes grandes y 2\% para lotes peque\~{n}os), al final se recolectaron alrededor de 700 muestras.
~\\Los resultados para este estudio indicaron que la mayor\'{i}a de las fuentes de yemas usadas por los viveristas estaban infectadas con el virus y se pudo apreciar c\'{o}mo el virus de la tristeza ha aumentado en un 80\% desde los \'{u}ltimos 10 a\~{n}os.


~\\\textbf{\citet{AC5}}
~\\Para este trabajo se tomaron muestras de hojas que presentaban s\'{i}ntomas de HLB en 4 viveros de Masaya, luego cada mes en cada vivero se tomaron 20 pl\'{a}ntulas de las plantas que presentaban s\'{i}ntomas en sus hojas, en total al final del muestreo cada vivero tiene 80 pl\'{a}ntulas muestreadas.
~\\Se analiz\'{o} la enfermedad presente en las plantas y se lleg\'{o} a la conclusi\'{o}n de que el porcentaje de infecci\'{o}n por vivero era de 12.5\%, 25\%, 12.5\% y 25\% respectivamente.

\subsection{Antecedentes estad\'{i}sticos}
~\\\textbf{\citet{AE1}}
~\\El objetivo fue evaluar el incumplimiento del mantenimiento de la cateterizaci\'{o}n venosa de un hospital mediante LQAS. Se realiz\'{o} en las \'{a}reas quir\'{u}rgicas, hospitalizaci\'{o}n, UCI y urgencias de un hospital de Murcia durante el a\~{n}o 2002(3 cortes) evaluando 4 criterios. Se parti\'{o} de un est\'{a}ndar de cumplimiento del 95\% y un umbral m\'{i}nimo del 85\%, un error a=5\% y un error b=20\%, se calcul\'{o} un tama\~{n}o de muestra de 44 casos y el n\'{u}mero m\'{i}nimo de cumplimientos del protocolo de 39.
~\\Durante el primer y segundo corte se obtuvieron 39 casos adecuados a protocolo, siendo de 42 en el tercer corte, por lo tanto, los resultados mostraron la inexistencia de un problema de calidad en el protocolo estudiado.

~\\\textbf{\citet{AE2}}
~\\El objetivo fue determinar las \'{a}reas de baja cobertura vacunal en cinco ciudades de Bangladesh. El estudio se realiz\'{o} en dos sets o grupos. En el primero, el objetivo fue evaluar la cobertura de inmunizaci\'{o}n en las ciudades Chittangong, Khulna y Rajshahi; los lotes eran todos los barrios de la ciudad. Para este grupo se seleccion\'{o} un 85\% de ni\~{n}os totalmente inmunizados como umbral superior y 60\% como umbral inferior, un nivel de confianza del 80\% y se calcul\'{o} el tama\~{n}o de muestra utilizando los m\'{e}todos convencionales descritos \textbf{Lemeshow et al} encontrando que el tama\~{n}o de muestra  por lote era de 13 ni\~{n}os, y el n\'{u}mero de aceptaci\'{o}n ser\'{i}a de 9 ni\~{n}os. En el segundo set, el umbral superior fue de 60\% e inferior de 40\%, un nivel de confianza del 95\% y se calcul\'{o} el tama\~{n}o de muestra utilizando el manual de la OMS la t\'{e}cnica de calidad de lote encontrando que el tama\~{n}o de muestra era de 16 ni\~{n}os por lote.
~\\Se observ\'{o} que la cobertura vacunal con BCG(primera vacuna administrativa) fue aceptable en todos los lotes estudiados concluyendo que se tiene una alta cobertura vacunal con BCG.

~\\\textbf{\citet{AE3}}
~\\Se utiliz\'{o} un muestreo de aceptaci\'{o}n de lotes para determinar la mejora en la adecuaci\'{o}n de ingreso y estancia en medicina interna, esto se logr\'{o} creando un umbral para el nivel de calidad aceptable usando como referencia la adecuaci\'{o}n que se ten\'{i}a antes de implementar el protocolo de mejora AEP. Se hicieron mediciones peri\'{o}dicas para evaluar el estado de la adecuaci\'{o}n de ingreso y estancia y se deten\'{i}a el protocolo si se detectaban cambios estructurales en el hospital, si esto ocurr\'{i}a se volv\'{i}an a las primeras fases del protocolo.

~\\La muestra es tomada cuando ingresa el paciente al hospital, luego se le hace un seguimiento durante todo el periodo que permanezca en el hospital y se le eval\'{u}an los indicadores de adecuaci\'{o}n (\% de ingreso adecuado y \% de estancia adecuada), tambi\'{e}n se tienen en cuenta las causas de ingreso inadecuado y estancia inadecuada. El lote son los ingresos y estancias en el SMI y se realiza cada 6 meses.

~\\Como resultados se obtuvieron para la primera evaluaci\'{o}n un ingreso adecuado del 83.7\% y una estancia adecuada del 46.6\%. Para la segunda evaluaci\'{o}n  el ingreso adecuado hab\'{i}a aumentado al 90.2\% y la estancia adecuada aument\'{o} al 64.4\%. De esto se concluye que se hab\'{i}a aumentado la calidad por medio del protocolo AEP y que el m\'{e}todo LQAS resulta ser \'{u}til como m\'{e}todo de evaluaci\'{o}n.


~\\\textbf{\citet{AE5}}
~\\En este libro, el autor realiza una comparaci\'{o}n bastante completa entre el muestreo de aceptaci\'{o}n y rechazo para atributos por el m\'{e}todo MIL-STD-105E/ANSI Z1.4 y los planes de muestreo c=0, dichas comparaciones se realizan en t\'{e}rminos de tama\~{n}os de muestra, curvas caracter\'{i}sticas de operaci\'{o}n y del AOQ(Calidad Promedio de Salida) y AOQL (M\'{a}ximo porcentaje de defectuosos esperado). Finalmente, se concluye que el m\'{e}todo propuesto en comparaci\'{o}n con el m\'{e}todo MIL-STD-105E/ANSI Z1.4 protegen a\'{u}n m\'{a}s al consumidor, siendo este m\'{e}todo mas estricto  a la hora de aceptar lotes con cantidades ``grandes'' de defectuosos. El m\'{e}todo por otra parte ``castiga'' al productor, ya que si este no tiene una calidad cercana al 100\%, tiende a rechazar muchos m\'{a}s lotes.
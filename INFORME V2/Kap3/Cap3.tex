\chapter{Metodolog\'{i}a}
~\\En esta secci\'{o}n se presenta la metodol\'{o}gia de la investigaci\'{o}n para dar soluci\'{o}n a la pregunta problema y a los objetivos planteados. Como primer punto se realiz\'{o} una visita a algunos viveros para tener una idea de la ubicaci\'{o}n, orden y proceso de producci\'{o}n de los lotes de plantas de c\'{i}tricos en campo, ya que es un factor de vital importancia para el proceso de simulaci\'{o}n y validaci\'{o}n de las propuestas muestrales. Posterior a esto, como segundo punto se desarrollar\'{a}n diferentes propuestas muestrales que se podr\'{i}an implementar seg\'{u}n la categorizaci\'{o}n de los lotes de acuerdo a su tama\~{n}o y se probar\'{a}n por medio de estudios de simulaci\'{o}n en distintos escenarios contextuales, tratando de abarcar todas las posibles situaciones que puedan ocurrir con el objetivo de evaluar cu\'{a}l de estas propuestas es la m\'{a}s \'{o}ptima. 

\section{Visita viveros}
Se visitaron dos viveros ubicados en Caicedonia - Valle del Cauca y Calarc\'{a} - Quind\'{i}o, los cuales presentaban una capacidad m\'{a}xima aproximada de 70000 y 120000 plantas respectivamente. Se utiliz\'{o} una gu\'{i}a de observaci\'{o}n de elaboraci\'{o}n propia (ver anexo 1), la cual ayud\'{o} en el proceso de extraer la informaci\'{o}n necesaria para aplicar la metodolog\'{i}a y para el respectivo estudio de simulaci\'{o}n, en esta gu\'{i}a de observaci\'{o}n se le pregunt\'{o} a los productores sobre algunos aspectos de inter\'{e}s, como por ejemplo, si hab\'{i}an realizado muestreos anteriormente y de qu\'{e} forma se hizo, qu\'{e} riesgo estaban dispuestos a asumir con el muestreo, qu\'{e} precauciones toma para evitar posibles infecciones en las plantas, entre otras.  

\section{Calculo de los tama\~{n}os de muestra para cada tama\~{n}o de lote (Metodolog\'{i}a AoZ)} 

Se har\'{a} uso de la metodolog\'{i}a de los panes $c=0$ o ``aceptar en ninguno'' para calcular los tama\~{n}os de muestra correspondientes a cada lote para todos los posibles tama\~{n}os que estos podr\'{i}an llegar a tener.

\section{Categorizaci\'{o}n de lotes} 

Se categorizar\'{a}n los lotes en rangos seg\'{u}n la cantidad de plantas, agrupando los tama\~{n}os de lotes para los cuales el tama\~{n}o de muestra sigue siendo el mismo, es decir, se establecer\'{a}n rangos de cantidades de plantas dentro del cual el tama\~{n}o de muestra no var\'{i}a o su variaci\'{o}n es insignificante. Por ejemplo, si formamos un rango de 2000 a 3000 plantas, esto significa que si se tiene un lote de 2060 plantas el tama\~{n}o de muestra que se obtiene bajo la metodolog\'{i}a de los planes $c=0$ es exactamente el mismo o muy poco variable que si tuvi\'{e}ramos un lote de 2800 o 3000 plantas.

\section{Dise\~{n}o del plan de muestreo en campo}

Se dise\~{n}ar\'{a}n las propuestas metodol\'{o}gicas que se implementar\'{a}n en los viveros con el fin de que la muestra recogida no solo sea aleatoria sino tambi\'{e}n lo m\'{a}s representativa posible, se utilizar\'{a} el muestreo aleatorio simple (MAS), el muestreo sistem\'{a}tico (MSIS) y muestreo por transectos, finalmente se analizar\'{a} cu\'{a}l de estas t\'{e}cnicas representa mejor la poblaci\'{o}n (lote), sin dejar de lado la facilidad log\'{i}stica que implica usar uno u otro en la pr\'{a}ctica.


\section{Calculo de los indicadores de desempe\~{n}o (AOQ, AOQL, ATI, OC)}

Una vez dise\~{n}ado el plan de muestreo en campo y definido qu\'{e} tama\~{n}os de muestra se van a utilizar, pasamos a evaluar el dise\~{n}o muestral, esto se realiza a partir de los indicadores de desempe\~{n}o y sus respectivas gr\'{a}ficas. Cada una nos dir\'{a} qu\'{e} tan bueno es el plan de muestreo implementado, qu\'{e} riesgos se llegan a correr en cuanto a la probabilidad de aceptar lotes con calidades inaceptables (error tipo II - riesgo del consumidor $\beta$) y rechazar lotes con calidades aceptables (error tipo I - riesgo del productor $\alpha$) , tambi\'{e}n se puede observar la calidad promedio de salida y la cantidad de plantas del lote que se deber\'{i}an inspeccionar con el plan de muestreo implementado.

\section{Categorizaci\'{o}n de los viveristas seg\'{u}n el nivel de riesgo} 

Luego de tener todo lo respectivo al muestreo, se planea crear 3 niveles de riesgo, flexible, normal y riguroso, esto para tener en cuenta en el muestreo el historial de los viveristas y ``premiar'' a los que mantengan una calidad alta en su producci\'{o}n, as\'{i} pues los viveristas comenzar\'{a}n en un nivel de riesgo normal, y pasaran a flexible o riguroso dependiendo de la cantidad de lotes ``buenos o malos'' que tengan en su historial, esto implica que los que tengan una calidad ``buena'' en sus viveros tendr\'{a}n tama\~{n}os de muestra m\'{a}s peque\~{n}os y por el contrario, los que tengan calidad ``mala'' (varios lotes rechazados) pasar\'{a}n a tener un  tama\~{n}o de muestra mayor, lo que implica m\'{a}s costos.

\section{Validaci\'{o}n de los planes por medio de simulaci\'{o}n}

Por \'{u}ltimo se validar\'{a}n los planes dise\~{n}ados utilizando herramientas computacionales, donde se simular\'{a}n todos los posibles escenarios de los viveros con los diferentes planes para as\'{i} conocer si estos cumplen con el objetivo de detectar lotes infectados con una confianza alta.



\chapter{Metodolog\'{i}a}
~\\En esta secci\'{o}n se presenta la metodol\'{o}gica de la investigaci\'{o}n para dar soluci\'{o}n a la pregunta problema y a los objetivos planteados. Como primer punto analizamos las caracter\'{i}sticas que presentan las unidades muestrales teniendo en cuenta sus tipos y c\'{o}mo estas se encuentran distribuidas en los lugares de muestreo. Posterior a esto, como segundo punto se desarrollar\'{a}n diferentes propuestas muestrales que se podr\'{i}an implementar y se probar\'{a}n por medio de estudios de simulaci\'{o}n en distintos escenarios contextuales, tratando de abarcar todas las posibles situaciones que puedan ocurrir con el objetivo de evaluar cu\'{a}l de estas propuestas es la m\'{a}s \'{o}ptima. 

\section{Visita viveros}
Se visitaron dos viveros ubicados en Caicedonia - Valle del Cauca y en Calarc\'{a} - Quind\'{i}o, los cuales presentaban una capacidad m\'{a}xima aproximada de 70000 y 120000 plantas respectivamente. Se utiliz\'{o} una gu\'{i}a de observaci\'{o}n de elaboraci\'{o}n propia, la cual ayud\'{o} en el proceso de extraer la informaci\'{o}n necesaria para aplicar la metodolog\'{i}a y para el respectivo estudio de simulaci\'{o}n.

\section{Categorizaci\'{o}n de lotes} 

Para este punto se categorizar\'{a}n los lotes seg\'{u}n su tama\~{n}o con el fin de conocer qu\'{e} plan aplicar para cada tama\~{n}o del lote ($N$), esto es qu\'{e} tama\~{n}o de la muestra tomar cuando el lote es peque\~{n}o, mediano o grande; las categor\'{i}as se definir\'{a}n con rangos de tama\~{n}os.

\section{Dise\~{n}o del plan de muestreo en campo}

Para este punto se dise\~{a}ar\'{a}n las propuestas metodol\'{o}gicas que se implementar\'{a}n en los viveros con el fin de que la muestra recogida no solo sea aleatoria si no tambi\'{e}n representativa.

\section{Calculo de los tama\~{n}os de muestra para cada categor\'{i}a (Metodolog\'{i}a AoZ)} 

Se har\'{a} uso de la metodolog\'{i}a de los panes $c=0$ para calcular los tama\~{n}os de muestra correspondiente en cada categor\'{i}a previamente definida, esto con el fin de que cada rango de tama\~{n}o tenga un respectivo tama\~{n}o de muestra o porcentaje de plantas a muestrear.

\section{Calculo de los indicadores de desempe\~{n}o (AOQ, AOQL, ATI, OC)}

Una vez dise\~{n}ado el plan de muestreo y definido qu\'{e} tama\~{n}os de muestra se van a utilizar, pasamos a evaluar el dise\~{n}o muestreal, esto lo hacemos a partir de los indicadores del dise\~{n}o y sus respectivas curvas. Cada una nos dir\'{a} qu\'{e} tan bueno es el muestreo, qu\'{e} riesgos se llegan a correr en cuanto a la probabilidad de aceptar lotes con $x$ cantidad de plantas infectadas, tambi\'{e}n la calidad promedio de salida y la cantidad de plantas que se deber\'{i}an inspeccionar con dicho plan si existen $x$ cantidad de plantas infectadas en el lote.

\section{Categorizaci\'{o}n de los viveristas seg\'{u}n el nivel de riesgo} 

Luego de tener todo lo respectivo al muestreo, se planea crear 3 niveles de riesgo, flexible, normal y riguroso, esto para tener en cuenta en el muestreo el historial de los viveristas y ``premiar'' a los que mantengan una calidad en sus viveros alta, as\'{i} pues los viveristas comenzar\'{a}n en un nivel de riesgo normal, y pasaran a flexible o riguroso dependiendo de la cantidad de lotes ``buenos o malos'' que tengan en su historial, as\'{i} los que tengan una calidad ``buena'' en sus viveros tendr\'{a}n tama\~{n}os de muestra mas peque\~{n}os y por el contrario los que tengan calidad ``mala'' (varios lotes rechazados) pasaran a tener un  tama\~{n}o de muestra mayor, lo que implica m\'{a}s costos.

\section{Validaci\'{o}n de los planes por medio de simulaci\'{o}n}

Por \'{u}ltimo se validar\'{a}n los planes dise\~{n}ados utilizando herramientas computacionales donde, se simular\'{a}n todos los posibles escenarios de los viveros con los diferentes planes para as\'{i} conocer si estos cumplen con el objetivo de detectar lotes infectados con una confianza alta.



\chapter{Introducci\'{o}n}

\section{Planteamiento del problema}
~\\En Colombia existen 97.275 hect\'{a}reas sembradas de c\'{i}tricos entre cultivos de naranja, lim\'{o}n, mandarina, toronja, tangelo, pomelo y lima, seg\'{u}n datos del Ministerio de Agricultura y Desarrollo Rural, es el grupo de frutales con mayor \'{a}rea sembrada en el pa\'{i}s despu\'{e}s del pl\'{a}tano, y genera aproximadamente 413.374 empleos directos e indirectos.

~\\Seg\'{u}n la organizaci\'{o}n de las Naciones Unidas para la Alimentaci\'{o}n y la Agricultura (FAO), el territorio colombiano es una de las siete naciones que puede volverse despensa mundial de alimentos, gracias a que tiene suficiente tierra para ampliar la frontera agr\'{i}cola sin necesidad de talar bosques. Por otro lado, \'{e}ste goza de privilegios naturales como ser el tercer pa\'{i}s con mayores recursos de agua y con diversidad clim\'{a}tica. A pesar de las ventajas comparativas que ofrecen muchas regiones del pa\'{i}s para el desarrollo citr\'{i}cola, la falta de escalas comerciales significativas, la alta dispersi\'{o}n geogr\'{a}fica de la producci\'{o}n, la falta de gesti\'{o}n empresarial y de desarrollo tecnol\'{o}gico, hacen que la producci\'{o}n y comercializaci\'{o}n de c\'{i}tricos sean poco competitivos en el mercado nacional e internacional. Adem\'{a}s, el pa\'{i}s enfrenta problemas para incursionar en los mercados externos debido a que, entre otros factores, no se cuenta con las variedades ni calidades adecuadas requeridas, no hay continuidad en la oferta exportable e igualmente se deben superar problemas de empaque y presentaciones, as\'{i} como barreras t\'{e}cnicas y sanitarias. Inclusive, existe poco grado de integraci\'{o}n entre la industria y la agricultura, no hay material vegetal certificado, falta investigaci\'{o}n y transferencia de tecnolog\'{i}a (desarrollo de variedades y calidades) en la fase agr\'{i}cola y agroindustrial, as\'{i} como prevenci\'{o}n de plagas y enfermedades.

~\\Existen diversas enfermedades que afectan a los c\'{i}tricos transmitidas principalmente por injertaci\'{o}n, vectores (organismos o insectos), y uso de herramienta, las cuales son muy da\~{n}inas para este cultivo. Las enfermedades que se presentan con mayor frecuencia en Colombia y las m\'{a}s importantes son el virus de la tristeza, Huanglongbing(HLB), Leprosis y Exocortis; cada una de ellas posee caracter\'{i}sticas espec\'{i}ficas en cuanto a su sintomatolog\'{i}a y consecuencias, \'{e}stas debilitan el \'{a}rbol, generando producciones escasas o con un valor inferior al establecido, y en casos avanzados pueden llegar a matar el \'{a}rbol. Sin embargo, en el pa\'{i}s no se ha implementado o desarrollado un sistema de certificaci\'{o}n de material vegetal que garantice la calidad de la propagaci\'{o}n y la seguridad de la especie.

~\\El problema principal es que la mayor\'{i}a de estas enfermedades son asintom\'{a}ticas en la etapa de vivero (edades tempranas de la planta) que tiene una duraci\'{o}n de 12 a 36 meses, es decir, en esta etapa no se puede diferenciar a simple vista una planta infectada con una no infectada, por lo que se hace necesario aplicar una prueba serol\'{o}gica para saber el verdadero estado de la planta. Al sembrar una planta con alguna de estas infecciones desde el comienzo, se perder\'{i}a mucho dinero invirtiendo en su mantenimiento y no se obtendr\'{i}an las ganancias o productos esperados, por lo cual se necesita asegurar o garantizar que las plantas que van a ser sembradas y entregadas est\'{e}n limpias de \'{e}stas enfermedades, logrando de esta manera la producci\'{o}n de material certificado. Dado que para evaluar las plantas se debe realizar la prueba serol\'{o}gica DAS-ELISA, y los lotes de c\'{i}tricos por lo general tienen una poblaci\'{o}n considerablemente grande, es imposible realizar un censo a todos los lotes que van a ser entregados por log\'{i}stica y econom\'{i}a. Por lo que a partir de esto surge la pregunta: ?`Es posible dise\~{n}ar un plan de muestreo adecuado y asequible que permita la detecci\'{o}n temprana de \'{e}stas enfermedades en los c\'{i}tricos?


\section{Justificaci\'{o}n}
Nuestro rol como estad\'{i}sticos es de vital importancia para lograr verificar que la producci\'{o}n est\'{e} libre de cualquier plaga y mitigar en gran medida posibles p\'{e}rdidas en toda la industria por lotes infectados, logrando que productores y consumidores se ven beneficiados.

\section{Objetivos}
\subsection{Objetivo General}
\begin{itemize}
\item Dise\~{n}ar y validar un plan de muestreo para aceptaci\'{o}n y rechazo de lotes de c\'{i}tricos en viveros del Valle del Cauca que permita estimar la cantidad de plantas infectadas en el lote..
\end{itemize}
\subsection{Objetivos Espec\'{i}ficos}
\begin{itemize}
\item Proponer y dise\~{n}ar diferentes tipos de muestreo tipo aceptaci\'{o}n/rechazo para lotes de c\'{i}tricos en viveros del Valle del Cauca.
\item Validar los dise\~{n}os mu\'{e}strales por medio de simulaci\'{o}n teniendo en cuenta confianza y costo del muestreo.
\item Estimar la cantidad de plantas infectadas en el lote.
\end{itemize}
\section{Antecedentes}
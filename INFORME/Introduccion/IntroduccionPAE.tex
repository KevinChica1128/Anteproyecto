\chapter{Introducci\'{o}n}

\section{Planteamiento del problema}
~\\En Colombia existen 97.275 hect\'{a}reas sembradas de c\'{i}tricos entre cultivos de naranja, lim\'{o}n, mandarina, toronja, tangelo, pomelo y lima, seg\'{u}n datos del Ministerio de Agricultura y Desarrollo Rural, es el grupo de frutales con mayor \'{a}rea sembrada en el pa\'{i}s despu\'{e}s del pl\'{a}tano, y genera aproximadamente 413.374 empleos directos e indirectos.

~\\Seg\'{u}n la organizaci\'{o}n de las Naciones Unidas para la Alimentaci\'{o}n y la Agricultura (FAO), el territorio colombiano es una de las siete naciones que puede volverse despensa mundial de alimentos, gracias a que tiene suficiente tierra para ampliar la frontera agr\'{i}cola sin necesidad de talar bosques. Por otro lado, \'{e}ste goza de privilegios naturales como ser el tercer pa\'{i}s con mayores recursos de agua y con diversidad clim\'{a}tica. A pesar de las ventajas comparativas que ofrecen muchas regiones del pa\'{i}s para el desarrollo citr\'{i}cola, la falta de escalas comerciales significativas, la alta dispersi\'{o}n geogr\'{a}fica de la producci\'{o}n, la falta de gesti\'{o}n empresarial y de desarrollo tecnol\'{o}gico, hacen que la producci\'{o}n y comercializaci\'{o}n de c\'{i}tricos sean poco competitivos en el mercado nacional e internacional. Adem\'{a}s, el pa\'{i}s enfrenta problemas para incursionar en los mercados externos debido a que, entre otros factores, no se cuenta con las variedades ni calidades adecuadas requeridas, no hay continuidad en la oferta exportable e igualmente se deben superar problemas de empaque y presentaciones, as\'{i} como barreras t\'{e}cnicas y sanitarias. Inclusive, existe poco grado de integraci\'{o}n entre la industria y la agricultura, no hay material vegetal certificado, falta investigaci\'{o}n y transferencia de tecnolog\'{i}a (desarrollo de variedades y calidades) en la fase agr\'{i}cola y agroindustrial, as\'{i} como prevenci\'{o}n de plagas y enfermedades.

~\\Existen diversas enfermedades que afectan a los c\'{i}tricos transmitidas principalmente por injertaci\'{o}n, vectores (organismos o insectos), y uso de herramienta, las cuales son muy da\~{n}inas para este cultivo. Las enfermedades que se presentan con mayor frecuencia en Colombia y las m\'{a}s importantes son el virus de la tristeza, Huanglongbing(HLB), Leprosis y Exocortis; cada una de ellas posee caracter\'{i}sticas espec\'{i}ficas en cuanto a su sintomatolog\'{i}a y consecuencias, \'{e}stas debilitan el \'{a}rbol, generando producciones escasas o con un valor inferior al establecido, y en casos avanzados pueden llegar a matar el \'{a}rbol. Sin embargo, en el pa\'{i}s no se ha implementado o desarrollado un sistema de certificaci\'{o}n de material vegetal que garantice la calidad de la propagaci\'{o}n y la seguridad de la especie.

~\\El problema principal es que la mayor\'{i}a de estas enfermedades son asintom\'{a}ticas en la etapa de vivero (edades tempranas de la planta) que tiene una duraci\'{o}n de 12 a 36 meses, es decir, en esta etapa no se puede diferenciar a simple vista una planta infectada con una no infectada, por lo que se hace necesario aplicar una prueba serol\'{o}gica para saber el verdadero estado de la planta. Al sembrar una planta con alguna de estas infecciones desde el comienzo, se perder\'{i}a mucho dinero invirtiendo en su mantenimiento y no se obtendr\'{i}an las ganancias o productos esperados, por lo cual se necesita asegurar o garantizar que las plantas que van a ser sembradas y entregadas est\'{e}n limpias de \'{e}stas enfermedades, logrando de esta manera la producci\'{o}n de material certificado. Dado que para evaluar las plantas se debe realizar la prueba serol\'{o}gica DAS-ELISA, y los lotes de c\'{i}tricos por lo general tienen una poblaci\'{o}n considerablemente grande, es imposible realizar un censo a todos los lotes que van a ser entregados por log\'{i}stica y econom\'{i}a. Por lo que a partir de esto surge la pregunta: ?`Es posible dise\~{n}ar un plan de muestreo adecuado y asequible que permita la detecci\'{o}n temprana de \'{e}stas enfermedades en los c\'{i}tricos?


\section{Justificaci\'{o}n}
Nuestro rol como estad\'{i}sticos es de vital importancia para lograr verificar que la producci\'{o}n est\'{e} libre de cualquier plaga y mitigar en gran medida posibles p\'{e}rdidas en toda la industria por lotes infectados, logrando que productores y consumidores se ven beneficiados.

\section{Objetivos}
\subsection{Objetivo General}
\begin{itemize}
\item Dise\~{n}ar y validar un plan de muestreo para aceptaci\'{o}n y rechazo de lotes de c\'{i}tricos en viveros del Valle del Cauca que permita estimar la cantidad de plantas infectadas en el lote.
\end{itemize}
\subsection{Objetivos Espec\'{i}ficos}
\begin{itemize}
\item Proponer y dise\~{n}ar diferentes tipos de muestreo tipo aceptaci\'{o}n/rechazo para lotes de c\'{i}tricos en viveros del Valle del Cauca.
\item Validar los dise\~{n}os mu\'{e}strales por medio de estudios de simulaci\'{o}n.
\item Estimar la cantidad de plantas infectadas en el lote.
\end{itemize}
\section{Antecedentes}
~\\A continuaci\'{o}n se muestran algunos antecedentes tanto contextuales como estad\'{i}sticos. El primer grupo de investigaciones, son estudios sobre an\'{a}lisis de estas enfermedades en lotes de plantas, a pesar de que no tienen un an\'{a}lisis estad\'{i}stico, han sido de gran ayuda, ya que se aplicaron diferentes tipos de muestreo para la evaluaci\'{o}n de estas enfermedades, y nos permiten tener una idea de la distribuci\'{o}n de las plantas en los lotes y porcentajes de infecci\'{o}n para realizar nuestras correspondientes simulaciones. El segundo grupo de investigaciones, corresponde a estudios fuera de nuestro contexto de inter\'{e}s, en los cuales se aplic\'{o} el muestreo de aceptaci\'{o}n y rechazo en distintas problem\'{a}ticas, a pesar de que todos estos estudios fueron en el \'{a}rea de la salud, son muy \'{u}tiles ya que nos centramos en analizar el funcionamiento de esta herramienta estad\'{i}stica para posteriormente llevarla a nuestro contexto.

~\\\textbf{\citet{AC1}}
~\\El objetivo de esta tesis fue el estudio de los distintos factores que determinan la epidemiolog\'{i}a de PPV y CTV en vivero, con el fin de establecer posibles estrategias de control. Se utiliz\'{o} un muestreo que consist\'{i}a en dividir las parcelas en bloques estad\'{i}sticos imaginarios(dos bloques de plantas) y finalmente se tomaron ciertas filas de plantas(4 primeras filas).

~\\\textbf{\citet{AC2}}
~\\El objetivo de este trabajo fue estudiar las enfermedades causadas por las especies de Phytophthora. Se exploraron 23 viveros de plantas ornamentales, muestreando solamente plantas sintom\'{a}ticas, analizando un total de 360 plantas.

~\\\textbf{\citet{AC3}}
~\\El objetivo fue describir la estructura de las comunidades de Phytophthora en 4 viveros comerciales. Se tomaron muestras de los 4 viveros cada 2 meses durante 4 a\~{n}os, recolectando 5 plantas de cada género en cada fecha de muestreo. Se seleccionaron plantas sintom\'{a}ticas.

~\\\textbf{\citet{AC4}}
~\\Para este proyecto se tomaron muestras de 9 viveros y dentro de cada vivero se tomaron muestras del 1\% para lotes grandes y 2\% para lotes peque\~{n}os. Los lotes estaban en rangos de entre 2000 y 40000 plantas, se determino que el CTV increment\'{o} en m\'{a}s del 80\% desde los \'{u}ltimos 10 a\~{n}os.

~\\\textbf{\citet{AC5}}
~\\Para este experimento se tomaron pl\'{a}ntulas de plantas en las cuales sus hojas presentaban s\'{i}ntomas de HLB, cada mes se tomaron 20 pl\'{a}ntulas, al final se ten\'{i}an 80 pl\'{a}ntulas por vivero. Los resultados obtenidos fueron que para cada vivero la proporci\'{o}n de plantas infectadas con HLB era del 12.5\%, 25\%, 12.5\% y 25\% respectivamente.

~\\\textbf{\citet{AE1}}
~\\El objetivo fue evaluar el incumplimiento del mantenimiento de la cateterizaci\'{o}n venosa de un hospital mediante LQAS. Se evaluaron 4 criterios. Se parti\'{o} de un est\'{a}ndar de cumplimiento del 95\% y un umbral m\'{i}nimo del 85\%, un error a=5\% y un error b=20\%, se calcul\'{o} un tama\~{n}o de muestra de 44 casos y el n\'{u}mero m\'{i}nimo de cumplimientos del protocolo de 39.

~\\\textbf{\citet{AE2}}
~\\El objetivo fue determinar las \'{a}reas de baja cobertura vacunal en cinco ciudades de Bangladesh. En el primero, se seleccion\'{o} la meta del 85\% de cobertura como umbral superior y 60\% como umbral inferior, un nivel de confianza del 80\% y se encontr\'{o} que el tama\~{n}o de muestra era de 13, y el n\'{u}mero de aceptaci\'{o}n ser\'{i}a de 9 ni\~{n}os. En el segundo, el umbral superior fue de 60\% e inferior de 40\%, un nivel de confianza del 95\% y se encontr\'{o} que el tama\~{n}o de muestra era de 16 ni\~{n}os.

~\\\textbf{\citet{AE3}}
~\\En este proyecto se utiliz\'{o} el m\'{e}todo de LQAS para evaluar la calidad, se tomaron muestras de pacientes en diferentes periodos primero evaluando la calidad del ingreso y estancia actual y luego tras hacer ciertas mejoras. Crearon un umbral de mala calidad que es ``\% de adecuaci\'{o}n" que funciona como regla de parada.

~\\\textbf{\citet{AE4}}
~\\Este proyecto utiliza el m\'{e}todo de LQAS para clasificar ni\~{n}os que padecen de tracoma en 3 tres categor\'{i}as de prevalencia, se crearon umbrales para determinar las categor\'{i}as y bas\'{a}ndose en datos de encuestas se decide si un ni\~{n}o pertenece a una categor\'{i}a o a otra.
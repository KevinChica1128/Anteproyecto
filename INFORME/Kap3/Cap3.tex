\chapter{Metodolog\'{i}a}
~\\En esta secci\'{o}n se presenta la propuesta metodol\'{o}gica para dar soluci\'{o}n a la pregunta problema. Como primer punto vemos las caracter\'{i}sticas que tienen las unidades que vamos a muestrear teniendo en cuenta sus tipos y c\'{o}mo estas se encuentran distribuidas en los lugares de muestreo. posterior a esto como segundo punto se desarrollar\'{a}n diferentes propuestas muestrales que se podr\'{i}an implementar y se probar\'{a}n por medio de estudios de simulaci\'{o}n en distintos escenarios contextuales, tratando de abarcar todas las posibles situaciones que puedan ocurrir con el objetivo de evaluar cu\'{a}l de estas propuestas es la m\'{a}s \'{o}ptima. Los pasos a realizar se resumen en los siguientes items

\begin{itemize}
\item Caracterizar las unidades de muestreo.
\item Catalogar los viveros y las enfermedades.
\item Generar diferentes propuestas para planes de muestreo.
\item Establecer diferentes escenarios contextuales.
\item Simular las diferentes propuestas de planes de muestreo en los distintos escenarios contextuales.
\item Comparar la eficacia o la capacidad de detecci\'{o}n de los muestreos simulados, obteniendo el ``mejor'' o los ``mejores'' tipos de muestreo. 
\end{itemize}